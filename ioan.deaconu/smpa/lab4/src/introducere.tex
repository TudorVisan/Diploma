Componenta cu cea mai rapida evolutie dintr-un sitem de calcul, placa video datoareaza acest lucru
industriei jocurilor video.

Placa video, ca orice alta componenta periferica din calculator, a aparut din nevoia de a avea o
puere de procesoare cat mai mare. Prima placa video din lume, creata de IBM in 1981, avea o singura functionalitate,
afisarea a 25 de linii a cate 80 de caractere fiecare. Desi nu pare un lucru avansat, la vremea
respectiva era un progres tehnologic imens deoarece a mutat randarea imaginilor de pe procesor
catre un echipament periferic dedicat. Urmatorul pas a fost randarea de imagini, lucru permis de
catre iSBX 275 Video Graphics. Aceasta placa video, aparuta in 1983, suporta rezolutii de pana la
256x256 pixeli si 8 culori. 

Aceasta placa video a permis aparitia jocurilor video pentru PC. Industria jocurilor a evoluat
rapid, cerand placi video din ce in ce mai rapide, astfel in 1995, au aparut primele placi video
capabile sa accelereze hardware randarea 3D. Desi erau capabile de randare 3D, fiecare producator
avea un api 3D diferit de ceilalti producatori, iar din aceasta cauza, Microsoft a lansat un api
comun in anul 1995, numit DirectX. 



