% ********** Sparrow protocol **********

\chapter{Sparrow Protocol}

\todo{intro}

As mentioned earlier, the main focus of this protocol is to increase network
bandwidth and responsiveness. This is achieved through larger windows of
activity of nodes closer to the gateway. 

\section{Initialization and registering into the network}

After power-on all Sparrow nodes will first read their pre-programmed id from
EEPROM memory and then attempt to register into a network. No restrictions such
as network address or parent id are imposed so any distribution of Sparrow
nodes where all leaf nodes are within communication distance of at least one
root node and a path can be formed from any root node to the gateway will
eventually form a network. 

There are two major approaches to registering into a network: the new node can
request registration information from the existing nodes or the existing nodes
can periodically advertise this information for new nodes. 

The first method implies that any new node must broadcast a discovery message,
wait for responses, choose the best parent node from the offers received and
finally register to that node. With this method registration can be
unsuccessful for five reasons. Firstly, there may be no root nodes within
communication distance of the new node, however, as the node communication
distance can be calculated beforehand and cannot be affected by the protocol,
the most we can do is set a timeout and, if this timeout is reached and no
offer is received, place the node in a sleep state and retry the procedure
after a certain amount of time. Secondly, there may be root nodes in range but
none are awake or in listening state. This can be partially solved by
prolonging root node listening periods, albeit to the detriment of power
consumption. Thirdly, there may be root nodes in listening mode and in range
but the discovery message may be corrupted by an already scheduled
transmission. The possibility of this happening can be reduced by spacing
transmissions evenly within the root nodes listening period. Fourthly,
considering that the discovery message was successfully received, the offer
message may be corrupted because two nodes attempt to respond at the same time.
This means that a contention period must precede any response, lengthening the
registering node's listening window and requiring multiple attempts on the
behalf of the root nodes in order to respond. Lastly, considering that the
discovery message was received and there was at least one offer, the following
message sequence, i.e. the actual registration process, may be corrupted by a
scheduled transmission or other phenomena and the process will have to be
repeated.

The second method involves all root nodes periodically transmitting a beacon,
or Router Advertisement, that contains all the information necessary to
register into the network. New nodes must listen for these messages and then
communicate with their chosen parent in order to register into the network.
There are X issue that we must address with this procedure. Firstly, there may
be no root nodes in range of the new node so, as with the first method, we can
set a timeout for the registration process. Secondly, although there is at
least one root node in range, the request may interfere with a scheduled
transmission. To avoid this, we schedule a registration slot for each root node
in which it listens for incoming request and no other communication is
scheduled. Thirdly, although the request was sent within the root nodes
registration slot, it was corrupted by another node attempting to register to
with the same node. This can be resolved by adding a contention slot before the
registration. A node must first listen for a random period of time and then
successfully transmit a message to the root node in the contention period in
order to begin the registration process. Finally, there is no guarantee that
there will be a victor after the contention process. Sparrowv3 nodes have a 17
μs transition delay between listening and transmitting states which means that
if two nodes are offset by less than this they cannot avoid each other and a
collision will happen. This cannot be resolved because two nodes may choose the
same listening period randomly.

As can be seen from the comparison the first method implies longer listening
periods for both the registering node and existing root nodes. It also scales poorly
with the number of nodes in the network as the probability of interfering with
scheduled transmissions increases. This means that as the network grows it will
become increasingly difficult for new nodes to join. The second method implies
a longer listening period just for registering nodes and just during the
registration process. It also scales relatively well with the number of nodes
in the network. If all contentions have a winner then the mean waiting time to
join into the network in \emph{n} nodes attempt to register with the same root
node is 

\[wait_{avg} = \frac{n-1}{2}\]

\section{Leaf node operation}

\section{Root node operation}

\section{Fault conditions and solutions}

% ********** End of Sparrow protocol **********
