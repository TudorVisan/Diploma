\normalfont\normalsize
\chapter{Related Work}

Research has been done in the area of wireless sensor networks with mobile gateways, but very few include the same constraints as in our networks. Such constrains consits of long distances between nodes, simple, fast implementation and the posibility to run 

Where most research focuses on routing packets to a mobile sink, we consider applications where network lifetime and low-maintenance are more important, and as such energy consumption on the nodes must be kept to a minimum (in order to have 3+ years of network lifetime
with no maintenance/battery changes). Shifting the energy consumption to the sink is a direct consequence of this decision, under the assumption that there exists a flight path that allows the gateway to collect information from all the nodes.

\section{Standard WSN Protocols}

The protocols implemented in Wireless Sensor Network are based on sourounding node discovery in order to build a topology and find the best way they can multihop data to the gateway. This approach works best in a static environment, but in a dynamic environment or an eviroment were the distance between nodes is too big or the time between two data packets is too big, the network convergence will be slow or not even possible.

\section{Crop Monitoring}

A reasearch of using a drone for crop monitoring has been conducted at a vineyard. Their system was comprised of a unmanned quadcopter, an Arduino board with a GPRS module for long distance communication with  the drone and ZigBee and Crossbow’s TelosB as wireless sensing nodes. The drone was not controlled via the long-distance link, but through a Spektrum DX7SE 2.4 GHz remote control.

They demonstrated that a preprogramed UAV can be used to monitor multiple crops where a standard WSN could not be deployed because of the unique constrains imposed by the environment.

The cost of the implementation was relatively high compared to ours, the remote is 300\$, the same as the entire drone that we propose and the TelosB is 99\$. This data suggests that for their experiment the drone, communication module and the remote control were half the cost of the equipment.

Another problem was that they were not saving the data localy, but sending it back to the base station where it was proccesed and saved. This can represent a problem because the system cannot function propely unless a base station is supplied.

\section{Aware platform}

The Aware platform, proposed by Ays. Egül Tüysüz Erman, Lodewijk Van Hoesel and Paul Havinga from University of Twente, is a platform that integrates WSNs, UAVs, and actuators into a disaster response setting and provides facilities for event detection, autonomous network repair by UAVs, and quick response by integrated operational forces\cite{erman2008enabling}.

They use multiple UAVs to deploy new nodes that will replace the damaged ones and check if they function. The entire system still relies on a sink to collect the data and to send them to a base station.
