% ********** Conclusions and future work **********

\chapter{Conclusions and future work}

This thesis proposes a new communication protocol for Wireless Sensor Networks.
It increases network bandwidth and responsiveness by lengthening the duty-cycle
of nodes closer to the gateway. The increase in duty-cycle, and implicitly
power-consumption, is compensated through the use of a new type of node capable
of harvesting and storing energy from the surrounding environment. Network
lifetime is preserved by dividing of the network nodes into two
categories. The first, leaf nodes, run at low duty-cycles to minimize power
consumption. The second, root nodes, are capable of harvesting additional power
from alternative sources and adjust their activity according to their number of
children and proximity to the gateway.  Longer periods of activity enable
information from leaf nodes to reach the gateway in a single network period.
This allows for good network responsiveness and high throughput.

\section{Future work}

We plan to develop this protocol further, taking advantage of a new generation
of Sparrow nodes, Sparrowv4. Apart from the improved hardware platform a useful
addition to the protocol would be the possibility to have multiple data sinks.
This would increase the network tolerance to defects and potentially allow for
better power consumption for root nodes as the routing tree depth would
decrease. 

Another direction of research would be the possibility of transmiting data in
both directions through the network. This would make the protocol applicable to
Wireless Sensor and Actuator Networks (WSAN), greatly increasingly the fields
of application.

% ********** End of Conclusions and future work **********
