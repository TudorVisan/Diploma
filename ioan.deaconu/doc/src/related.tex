\normalfont\normalsize
\chapter{Related Work}
\label{chap:related}

% Truism
Previous research into applications of WSNs in remote locations have proposed mobile gateways mounted on UAVs. The main focus of that research is on communication protocols and data collecting, with a lower empahsis on costs. Different systems are proposed for integrating an UAV with a WSNs, but their application to real life scenarios is limited by the high costs of the equipment used and the necessary knowledge to install and operate that equipment.
The general directions of previous research into WSN and UAV integration are:
\begin{itemize}

\item Using node signals to perform course corrections for dynamic navigation
\item Using drones for node deployment, to create, expand or fix problems in the network \cite{akyildiz2002wireless}
\item Data muling protocols

\end{itemize}

\section{Standard WSN Protocols}

% am corectat un pic, dar tot nu prea cred ca sectiunea asta ar trebui sa existe
The protocols used in common Wireless Sensor Network deployments are based on neighboring node discovery in order to build a network topology and find the best multi-hop route to the gateway. This approach works best in static environments, but in a dynamic environment or an environment were the distance between nodes is large or the time between two data packets is long, network convergence is slow or even impossible.

\section{UAV experiments with Wireless Sensor Networks}

In \cite{teh2008experiments} a solution consisting of ground nodes with pre-assigned GPS positions is proposed. An RC plane would perform course corrections after receiving the current GPS position from the nodes in order to calculate the best path for muling the data from the network.

The advantage of using a plane in this experiment is the longer range and higher speed that it can offer as opposed to a quad-copter or a similar design.  But the high speeds create the problem of maneuverability. The plane has a turning range of 400 meters while a quad-coter drone can turn on the spot.

\section{Crop Monitoring}

In \cite{valente2011air} a system which uses a drone for crop monitoring at a vineyard is proposed. The system was comprised of a unmanned quad copter, an Arduino board with a GPRS module (used for long distance communication with  the drone), ZigBee and Crossbow’s TelosB as wireless sensing nodes. The drone was not controlled via the long-distance link, but through a Spektrum DX7SE 2.4 GHz remote control.

The authors demonstrate that a preprogrammed UAV can be used to monitor multiple crops where a standard WSN could not be deployed because of the unique constrains imposed by the environment.

The cost of the implementation is relatively high compared to our solution: the remote control alone costs 300\$, the same price as the drone we propose, and a TelosB node costs 99\$, almost 3 times as much as a Sparrowv3 node.

Another disadvantage of the system is that the data is not saved locally, but sent back to a base station where it is processed and saved. This can pose a problem in remote environments, were a base station cannot be deployed, as the system cannot function properly without one.

\section{Aware platform}

The Aware platform \cite{ollero2007aware}, proposed by Ays. Egül Tüysüz Erman, Lodewijk Van Hoesel and Paul Havinga from University of Twente, is a platform that integrates WSNs, UAVs and actuators into a disaster response situation and provides facilities for event detection, autonomous network repair by UAVs and quick response by integrated operational forces.

They use multiple UAVs to check the correct functioning of nodes and deploy new nodes that can replace damaged ones. The entire system still relies on a sink to collect the data and to send it to a base station \cite{erman2008enabling}.
