\chapter*{Abstract}

\emph{A User-friendly, Innovative Layout and Routing Algorithm for Diagrams }

\vspace{\baselineskip}

Diagrams are used in a variety of fields such as FPGA design as a means to represent workflows and concepts and require dedicated algorithms which can produce easy to understand results. 
Available software is clunky, at times it can be really slow and will often result in layouts which are hard to read and understand. The main issue of layout algorithms is the sheer amount 
of data that a diagram (graph) can contain, which makes placing and routing in a limited space extremely difficult. We attempted to solve this problem by using a two phase software which tries 
to generate layouts which feel natural for the user. The first phase of the software will layout the diagrams using a genetic algorithm and constraints provided by the user, while the second 
phase will route the paths connecting the diagrams using orthogonal routing. Using this approach, we can provide layouts which suit the needs and preferences of any user, instead of providing 
generic representations.

\vspace{\baselineskip}
\textbf{Keywords} 

\chapter*{Acknowledgements}

Mulțumiri 
