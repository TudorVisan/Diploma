Wireless Sensor Networks (WSNs) are being used more and more in almost every
field imaginable. Applications generally involve collection and processing of
data, either on-site or remote. Applications fields include home automation,
agriculture, military, space exploration etc.\cite{baggio2005wireless} In
order to collect the data, gateway platforms \cite{steenkamp2009wireless} are
required and most of them are stationary bulky devices or PCs connected to one
of the wireless nodes that serve as a base-station. Deployment in more remote
locations with these gateways is very difficult at best and impossible at
worst. 

We aim to show in this paper that there exists a solution for a truly mobile
gateway design that can be used either for collecting data or for debugging
large wireless network infrastructures. Our solution involves a system that
includes a high-mobility device (such as a quadcopter drone) and a WSN island
comprised of nodes that collect data, but are able to store a limited amount
of data until the data can be forwarded to the mobile
gateway.\cite{valente2011air}

The assumptions we make for the WSN are that of an actual deployment of nodes:
We have a multitude of sensors platforms monitoring atmospheric conditions at
fixed time intervals, that maintain a very low duty cycle: the period of
activity, in which the node measures or transmits/listens on the radio, is
less than 0.1\%. This makes routing almost impossible, since the time it takes
the aircraft to leave the transmission range is much smaller than the time it
would take to propagate a package along one single hop. Routing is also
prohibitively difficult in this scenario because of high storage demands: if
an aircraft passes by once per day, every node has collected measurements
ready to forward that make up a large portion of the available memory of each
node. The gateway, on the other hand, can be a more sophisticated device (such
as the one we are using, running Linux and with plenty of storage for
measurements), which enables it to collect everything the nodes have to offer. 

The challenge in this scenario is establishing a communication link with every
sensor node, by being at the correct location at the right time. Our system
does not assume that the sensor platforms are in range of one another but can
handle the situation in which they do, by establishing individual communication
links with each one at every fly-by. This is less taxing on individual sensor
platforms, as they do not have to route packages. Our assumption for shifting
the energy usage towards the gateway takes into account that radio
communication is a very small percentage of the power used by an aircraft
while flying.

We will show an overview of the system architecture in Chapter
\ref{chap:arch}, the hardware and software implementation in Chapter
\ref{chap:impl} and results of using the gateway in Chapter
\ref{chap:results}. We also take into account work already done on the subject
in Chapter \ref{sec:related}.

