\normalfont\normalsize
\chapter{Introduction}

Technology has been present in our lives for a long time and we have become so accustomed to it that we do not even realize when we are using it. This technological advance has lead to the development of small devices with communication capabilities and low power consumption. These devices can form a wireless sensor network which can be used to collect data from the environment and send it to a gateway for further processing.

%Ce vrei să spui cu asta? Ai putea să pui asta la începutul fiecărei lucrări de licență, dar tot nu spui nimic. 
% Wireless Sensor Networks are .... They allow us to sense phenomena in the environment and act upon them .. Applications would be X, Y, Z
% Some applications are hindered by the difficulty in obtaining the sensed data, due to long distances between WSN area and command center. We propose a system ... 
The applications of a Wireless Sensor Network are unlimited in number. For example they can be used to monitor crops, to detect possible forest fires, to detect the presence of animals or vehicles in certain areas, to track and monitor doctors in hospitals, to guide interactive toys, to detect and monitor car thefts etc. In the last ten years, integration of wireless sensor networks with unmanned aerial vehicles had been tested and proven to be successful. However, previous implementations, described in chapter \ref{chap:related}, are complicated, difficult to operate and too expensive for the general public.

One goal of a Wireless Sensor Network is to collect data and send it to a device called a gateway. A frequent scenario is that of a number of nodes deployed to monitor an area that is hard to reach for various reasons. Installation of a gateway in such an area is almost always impossible, but usually the area is accessible to an unmanned aerial vehicle. The way the UAV reaches that area can be either by being remote-controlled by a human operator or by following a list of predetermined waypoints specified by the user. When the UAV reaches the nodes, it can start retrieving and saving the data so it can be sent back home. The data collected by the drone should be accessible at any time, if the UAV is powered.
 
The solution that we propose is based on a very popular and easy to use drone, the AR Parrot Drone 2.0, and the Sparrow Family \cite{voinescu2013lightweight} of sensor nodes, developed at University POLITEHNICA of Bucharest by Andrei Voinescu.
% Nu trebuie plug așa mare în introducere, nu este relevant cine le-a făcut, cel mult lasă developed at UPB.
