Specificatiile de baza sunt:
\begin{itemize}

\item Hyper-threading.
\item Quad-core si octa-core nativ
\item Intel QPI
\item 64KB L1 cache pentru fiecare nucleu (32KB L1 data si 32KB L1 instruction)
\item 256KB L2 cache pentru fiecare nucleu
\item L3 cache de ordinul MB
\item Controler de memorie ram DDR3 integrat cu doua sau trei canale
\item A doua generate de Intel-VT
\item Instructiunea SSE4.2
\item Pipeline in 20 sau 24 de stagii.
     
      

\end{itemize}


Aceasta arhitectura are multe imbunatatiri care o ajuta sa fie relevanta si in aceste zile, dar
voi prezenta cele mai importante dintre ele.

\begin{itemize}

\item Accesul nealiniat la memoria cache. Pana la arhitectura Nehalem, accesul la memoria cache era
facut aliniat, astfel compilatoarele trebuiau sa se ocupe de alinierea memoriei pentru a obtine performanta
maxima. Cu access nealiniat la memoria cache, aplicatiile nu mai trebuie sa fie optimizate pentru
accesul nealiniat, marind viteza de executie a programului\cite{singhal2008inside}.

\item Hyper-Threading. Desi acest feature a fost folosit anterior in arhitectura Pentium 4 in anul
2003, s-a renuntat la el in urmatoarea generatie, pentru a se reveni inapoi la simultaneous threading odata cu arhitectura Core i7. 

\item Controlul Avansat al Consumului. Pe langa nuclee propriuzise ale procesorului, exista un
nucleu mai mic, un nucleu care se ocupa de monitorizarea parametrilor de functionare, parametri
precum temperatura, consumul , optimizand puterea totala consumata. Astfel, acest microcontroler
poate opri complet un nucleu pentru a minimiza consumul.

\item Modul Turbo Boost. Datorita controlului avansat de consum, si a faptului ca mai multe nuclee pot fi
oprite, aceasta rezerva de putere poate fi folosita pentru a mari frecventa nucleelor active. Cat
timp puterea totala consuma a procesorului si temperatura acestuia nu depasesc limitele admise,
exista posibilitatea ca toate nuclele sa poata functiona cu o frecventa marita in acelasi timp.

\end{itemize}

O mare parte din aceste idei au fost folosite in alte
arhitecturi mai recente, arhitecturi  precum Cortex A7, A9 sau mai recent A57 si A53. Deasemenea si
AMD a folosit o parte din idei in procesoare x86 create de ei.



