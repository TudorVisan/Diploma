\normalfont\normalsize
\chapter{Introduction}
 
Wireless Sensor Networks consist of small devices that can communicate with each other. They allow us to sense phenomena in the environment and act upon them. They can be used to monitor crops, to detect possible forest fires, to detect the presence of animals or vehicles in certain areas, to track and monitor doctors in hospitals etc.\cite{baggio2005wireless},\cite{arampatzis2005survey} 

One goal of a Wireless Sensor Network is to collect data and send it to a device called gateway. The gateway platform can be a PC connected to one of the nodes, a mobile phone or any device that can connect to a node and serve as a base-station.


Some applications are hindered by the difficulty in obtaining data from the nodes due to long distance between WSN area and base-station. Cable connection is not a possibility either but the area is accessible to an Unmanned Aerial Vehicle. The way the UAV is controlled in order to reach that area can be done by either using a remote control or using an autopilot that follows a list of predetermined waypoints specified by the user. As soon as the UAV reaches the nodes it can start retrieving and saving the data so it can be sent back home. The data collected by the drone should be accessible at any time, if the UAV is powered.
 
In the last ten years, integration of wireless sensor networks with unmanned aerial vehicles had been tested and proven to be successful. However, previous implementations, described in chapter \ref{chap:related}, are complicated, difficult to operate and too expensive for the general public.

The solution that we propose is based on a very popular and easy to use drone, the Parrot AR.Drone 2.0, and the Sparrow family \cite{voinescu2013lightweight} of sensor nodes, developed at University POLITEHNICA of Bucharest.
