% ********** Related work **********

\chapter{Related work}

There are many Medium Access Control (MAC) protocols that cover a multitude of
applications and scenarios. It is a broad research area however recently there
has been much work done in the area of low-power and wireless networks. MAC
protocols designed for Wireless Sensor Network can be broadly divided into two
categories: contention-based and TDMA-based protocols.

Contention-based MACs, such as the IEEE 802.11 Distributed Coordination
Function\cite{ieee1997wireless} (DCF), are used in ad-hoc wireless networks.
These protocols do not involve any synchronization between the nodes of the
network. They do however tend to have higher energy consumption because of idle
listening.

TDMA-based protocols require scheduling and the reservation of communication
slots. This is why they present a natural energy conservation advantage over
contention-based protocols as sleep periods can be increased and there is no
overhead caused by contention or collisions.

\section{S-MAC}

S-MAC\cite{ye2004medium} is a Medium Access Control protocol that attempts to
reduce power consumption by reducing collisions, overhearing, control overhead
and idle listening. It also aims to offer good scalability and collision
avoidance. It uses Request to Send/Clear to Send (RTS, CTS) signals in order to
reduce collisions and alternating listen/sleep periods to reduce energy
consumption. However, these improvements come at the price of latency. 

In order to reduce control overhead, neighbouring nodes are synchronized.
However, not all neighbouring nodes can be on the same schedule.  Figure
\ref{fig:s-mac_schedules} presents and example of two nodes, A and B, that
synchronize with nodes C, respectively D, which means the have different
schedules. This means they must use the RTS/CTS system to avoid causing a
collision. After nodes start a data transmission, they do no go back to sleep
until it has ended.

\begin{figure}[ht]
	\begin{center}
		\includegraphics{img/s-mac_schedules.pdf}
	\end{center}
	\caption{\small \itshape{Neighbouring nodes A and B with different
	schedules\protect\footnotemark}}
	\label{fig:s-mac_schedules}
\end{figure}
\footnotetext{Image taken from \emph{Medium access control with coordinated
adaptive sleeping for wireless sensor networks}\cite{ye2004medium}}

Synchronization between nodes is achieved by maintaining a schedule table that
stores the schedules of all its known neighbours. At startup nodes listen for a
schedule advertised by other nodes before choosing its own, randomly. If it
receives a schedule before choosing its own, it follows that schedule. If
however it arrive after a schedule is randomly chosen it adopts both schedules
but advertises its own one. 

Timer synchronization, in order to prevent counteract clock drift, is
accomplished through special SYNC packets.

\section{SCP}

SCP\cite{ye2006ultra} is another MAC protocol which is based on S-MAC. The main
problem it addresses are long preambles added to packets. Nodes function on
cyclic schedules of sleeping and listening. If a node does not detect another
transmission during the listening period it goes back to sleep, otherwise it
remains active to receive the transmission. This means that a node that wants
to send it a message must first send a signal during the destination node's
listening period. Because of clock drift this preamble can become quite long,
especially in the case of long synchronization periods, as shown in figure
\ref{fig:scp_long_preambles}.

SCP addresses this problem by maintaining a schedule for listening periods.
This means that a node can anticipate when the destination node will by
listening and can greatly reduce the preamble length, as shown in figure
\ref{fig:scp_short_preambles}.

Although scheduling does reduce preamble length it also add synchronization
overhead. This can be solved by adding synchronization information to data
packets, however this is not always possible. In the case that there is no data
packet to which to append the synchronization information a separate
\emph{sync} packet will be sent.

\clearpage

\begin{figure}[ht]
	\begin{center}
		\includegraphics[scale=0.9]{img/scp_long_preambles.jpg}
	\end{center}
	\caption{\small \itshape{Long preamble needed to transmit
	data\protect\footnotemark}}
	\label{fig:scp_long_preambles}
\end{figure}
\footnotetext{Image taken from \emph{Ultra-low duty cycle MAC with scheduled
channel polling}\cite{ye2006ultra}}

\begin{figure}[ht]
	\begin{center}
		\includegraphics[scale=0.9]{img/scp_short_preambles.jpg}
	\end{center}
	\caption{\small \itshape{Short preamble when using scheduled 
	listening\protect\footnotemark}}
	\label{fig:scp_short_preambles}
\end{figure}
\footnotetext{Image taken from \emph{Ultra-low duty cycle MAC with scheduled
channel polling}\cite{ye2006ultra}}

% ********** End of Related work **********
