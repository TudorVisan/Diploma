% ********** Hardware platform **********

\chapter{Hardware platform}

This protocol was developed and tested entirely on the Sparrowv3 wireless
sensing platform. This platform was developed by As. Drd. Ing. Andrei Voinescu
and Sl. Dr. Ing. Dan Tudose at the Faculty of Automatic Control and Computers,
University Politehnica of Bucharest. The development kit consists of Sparrowv3
sensing nodes and the SparrowDongle USB stick.

\begin{figure}[ht]
	\begin{center}
		\includegraphics[width=\textwidth]{img/sparrowv3_kit.jpg}
	\end{center}
	\caption{\small \itshape{The Sparrowv3 development kit}}
\end{figure}

Together, the sensing nodes and USB dongle form a tool for developing and
debugging applications that are IEEE 802.15.4 and ZigBee compatible. Such
applications range from environmental monitoring and data logging to wildlife
research.

\section{The Sparrowv3 nodes}

The Sparrowv3 wireless sensing nodes are small, battery powered devices capable
of IEEE 802.15.4 and ZigBee compatible communication. They are built around the
\mbox{ATmega128RFA1}, a microcontroller made by Atmel, which is based on the
\mbox{ATmega128}. It incorporates a powerful and energy-efficient AVR core and
comes equipped with all the standard AVR peripherals such as multiple 8 and
16-bit synchronous timers, an 8-bit asynchronous timer, SPI (Serial Peripheral
Interface), USART (Universal Synchronous and Asynchronous Receiver and
Transmitter) and TWI (Two Wire serial Interface) communication interfaces and
an ADC (Analog-to-Digital Converter). The controller offers an advanced power
management interface and sleep modes. This permits the user to deactivate
peripherals and put the CPU into various sleep modes in order to preserve
power. In its most efficient sleep state the microcontroller utilises only 0.25
uA of current (at a supply voltage of 3V). The controller is clocked from an
external 16 MHz crystal oscillator.

\begin{figure}[ht]
	\begin{center}
		\includegraphics[width=\textwidth]{img/sparrowv3_node.jpg}
	\end{center}
	\caption{\small \itshape{Sparrowv3 node architecture}}
\end{figure}

The \mbox{ATmega128RFA1} has 4 Kbytes of EEPROM memory used to hold
write-protected configuration and calibration data, 128 Kbytes of In-System
Self-Programmable Flash memory, which is used to store the firmware, and 16
Kbytes of SRAM, used for storing data during execution.

The microcontroller also incorporates a low-power 2.4 GHz transceiver and a MAC
symbol counter. The transceiver has a maximum sensitivity of -100 dBm and
includes a 128-bit hardware AES accelerator and a 2-bit true random number
generator. An antenna and a balun are to be connected to the controller in order
to ensure optimal radio performance, but a connector for an external antenna is
also present, to allow attaching a larger, more powerful antenna. The MAC
symbol counter is a 32-bit counter that provides the timing information for the
transceiver. The counter runs by deriving a 62.5 kHz clock from a 32.768 kHz
low-drift crystal oscillator. 

The Sparrowv3 sensing nodes are capable of measuring three aspects of their
surrounding environment: ambient light, temperature and humidity. The measuring
range for temperature is -40 to 125°C with an accuracy of 0.01°C. For humidity
the range is 0 to 100 \%RH with an accuracy of 0.04 \%RH. Ambient light can be
measured in the range of 0 to 75 Lux.

The node exposes both JTAG (Joint Test Action Group) and ISP (In-System
Programmer) programming connectors through an expansion header, along with an
USART connection and several GPIO (General Purpose Input/Output) lines. It can
also be powered through this header however supply voltage must be externally
regulated to 3.3 V or the board might be damaged.  The on-board power comes
from a CR2477N battery on the back of the board.

\section{The SparrowDongle}

The SparrowDongle is an USB interface device between an USB host and the 2.4 GHz
network. It is built around two microcontrollers: an ATmega128RFA1 and an
ATmega32U4. The first is responsible for interfacing with the wireless network,
receiving packets of data and delivering packets of data to the host. The
second is responsible for transferring these packets to the host via the USB
protocol. The ATmega32U4 is well suited for this role as it incorporates a full
speed USB controller. The communication between the two controllers is done via
a USART interface.

The board exposes ISP and JTAG interfaces for both controllers. It also has 4
LEDs for debugging purposes or displaying the current status.

\begin{figure}[ht]
	\begin{center}
		\includegraphics[width=\textwidth]{img/sparrowdongle.jpg}
	\end{center}
	\caption{\small \itshape{SparrowDongle architecture}}
\end{figure}

% ********** End of Hardware platform **********
