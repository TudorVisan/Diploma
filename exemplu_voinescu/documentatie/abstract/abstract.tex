% ********** Abstract **********
\chapter*{Abstract}

Wireless Sensor Networks are composed of various electronic devices with wireless communication capabilities denoted as nodes. They 
are used to build ``smart applications'' that benefit from large amounts of sensing data made available by these networks.

In the context of Wireless Sensor and Actuator Networks (WS\&ANs), obtaining sensor data is relatively easy with a
single-sink, single-hop approach, sensor data is transmitted to the gateway, which passes the data on to another network
(e.g. the Internet).

Heteregenous wireless sensor networks are a different matter. Nodes cannot be treated the same way as they do not
have the same capabilities, or types of sensors, actuators, energy sources and so on. Therefore a different approach is
needed, one that takes into account the differences between nodes as well as the purpose of the network.

A task-based system is more appropiate in the sense that a task exists for collecting data (e.g. a temperature
acquisition task), for processing data or for activating actuators. Such a task-based network requires however an entity
that assigns specific tasks to appropriate nodes, assuring minimal energy loss for each task that is run by the network,
maximizing network lifetime, as well as taking into consideration several other constraints, such as data dependencies,
minimizing radio communication, etc. 

The main contribution of this study is treating this problem as a partitioning problem of the tasks into groups, each group
representing one of the wireless nodes. We prove that the algorithm we propose is also very versatile and can include 
all of the constraints needed for energy-efficient wireless applications.

\textbf{Keywords} Wireless Sensor Networks, task, scheduling, graph cuts

\chapter*{Acknowledgements}

First I would like to express my gratitude to my advisor prof. dr. Nicolae \c{T}\u{a}pu\c{s} for his support in the SENSEI project.
This study has forked from work on that project and i am grateful that i could be part of the team.

I would also like to give special thanks to my co-advisor, Dan Tudose, for helping me overcome hardware difficulties, for
helping me correct this work and for great moral support. I am also very grateful to Emil Slu\c{s}anschi for the sum of great
feedback i've received from him for the early drafts and to R\u{a}zvan T\u{a}t\u{a}roiu for bouncing off ideas with me. This study would
not have been possible withouth them.

This research is partially supported by the European Union through the FP7 Programme (of which the SENSEI project is part of).

Last but not least I would like to say ``Thank you'' to my parents and girlfriend, who have shown understanding and have given me
the moral support needed.



% ********** End of Abstract **********
