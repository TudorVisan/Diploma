\normalfont\normalsize
\chapter{Hardware Platform}

In this chapter we will present the hardware platforms used. 

Because we wanted to emphasize not only a new way of acquiring data, but also a simple and low cost one, we have selected The Parrot AR.Drone 2.0 as the work horse that will carry the mobile gateway which communicates with the nodes from our Sparrow Family. The drone provides several key features that we require, such as a Linux embedded system, a mobile platform with the possibility of autonomous flying and sufficient flight time for our needs.

In order to keep the price low and a small footprint, the Sparrow Family has a surface mounted antenna on the PCB. The power of the antenna is 2 dBi and it is perfect for applications were size is a constraint. When the size can be overlooked, an external antenna can be mounted to extend the range of the device.

The cost of our solution depends on the configuration and size of the WSN islands. The cost is 300\$ for the drone, 35\$ for a node and 8 \$ for a high gain antenna. 

Example cost for two different types of networks:
\begin{itemize}

\item An area that needs a high density of nodes, but covers a small surface, does not require to nodes to have the external antenna. The cost would amount to 700\$   

\item  An area that needs a low density of nodes, but covers a larger surface, or are place in a high vegetation or interference area will require the nodes to be equipped with an external antenna. The cost would amount to 800\$

\end{itemize}

\section{The Parrot AR.Drone 2.0}\cite{parrot2012drone}

Parrot AR.Drone is a Wi-Fi radio controlled flying quad copter built by the French company Parrot.
The original drone was released in 2010 and in 2012 it was replaced by version 2.0. Since the launch of the original AR.Drone, more the half a million units have been sold, making it one of the, if not, the most popular drone on the market.\cite{parrotpopular}

The reason of its success is not entirely due to the relatively low price of around 300\$ but also because it is very easy to learn how to control the drone and also because of the USB port that accommodate any device using that interface and the Linux operating system, making it incredibly versatile and very easy to integrate in our system.


\begin{figure}[ht]
\begin{center}
\includegraphics[width=0.9\textwidth]{img/drone.jpg}
\end{center}
\caption{\small \itshape{The Parrot AR.Drone 2.0}\cite{parrot_drone}}
\end{figure}


Because of those reasons, the Drone has a number of after-market modules that can be attached to it, such as 
the Flight Recorder GPS Module. This module has a built-in storage of 4GB for video recording purposes and a built in GPS receiver. This allows the drone to follow a predetermined path of way points and to return back from where it took off automatically, all within the limit of the Wi-Fi connection with the control device.

In order to properly accommodate the Sparrow Dongle, the hull had to be carved. Also the required external antenna of the dongle was mounted on top of the polyester cover and a small counterweight  has been glued at the opposite side in which the dongle is located. The counterweight acts as a ballast that keeps the drone level.


\begin{figure}[ht]
\begin{center}
\includegraphics[width=0.9\textwidth]{img/counterweight.png}
\end{center}
\caption{\small \itshape{The counter weight need to balance the drone}}
\end{figure}

 

The Parrot AR.Drone 2.0 specifications are :
\begin{itemize}



\item   1GHz 32 bit ARM Cortex A8 processor with 800MHz video DSP TMS320DMC64x
\item   Linux 2.6.32
\item   1Gbit DDR2 RAM at 200MHz
\item   USB 2.0 high speed for extensions
\item   Wi-Fi b,g,n
\item   3 axis gyroscope 2000°/second precision
\item   3 axis accelerometer +-50mg precision
\item   3 axis magnetometer 6° precision
\item   Pressure sensor +/- 10 Pa precision
\item   Ultrasound sensors for ground altitude measurement
\item   60 fps vertical QVGA camera for ground speed measurement
\item	30 fps 720p front mounted camera 	

\end{itemize}


\section{The Sparrow Family}

\begin{figure}[ht]
\begin{center}
\includegraphics[width=0.9\textwidth]{img/sparrow.jpg}
\end{center}
\caption{\small \itshape{The Sparrow Dongle next to the SparrowV3.2}}
\end{figure}
 
The Sparrow Family, composed of the Sparrow Dongle and SparrowV3.2, use a 2,4 GHz wireless network as a medium of communication. 

The main component of this family is the ATMega128RFA1. It is an 8bit micro-controller from Atmel that has an on-chip 2.4 GHz wireless transceiver. On-chip transceivers occupy no extra space and require little extra electronics to operate, making the resulting boards very small. The on-chip transceiver allows more energy-efficient operating modes, and facilitate higher bandwidth transfers between the micro-controller's main memory and the transceiver frame-buffer, all
of which are important improvements in the field of wireless sensor networks. 

The signal received or sent by the wireless transceiver can be boosted by attaching an external antenna. For example, in an ideal situation, with no interferences from the outside world , an 8 dBi omni-directional antenna mounted on both communication devices would amount to an around 200 meters of communication range, well over the 70 meters measured with the default antennas.

\subsection{The Sparrow Dongle}

\begin{figure}[ht]
\begin{center}
\includegraphics[width=0.9\textwidth]{img/donge_architecture.png}
\end{center}
\caption{\small \itshape{SparrowDongle stick architecture}\cite{voinescu2013lightweight}}
\end{figure}
 
The link between the the wireless sensor networks and the rest of the digital world, the Sparrow Dongle is the gateway of the Sparrow Family. The Dongle can be connected to any device that has an USB port and can support USB CDC with ACM module (USB Communications Device Class with Abstract Control Mode). 

The dongle uses Atmega32U4 as a dedicated USB Controller unit. This design allows the RF controller to run any RF communication stack without having the USB code intrude on key timings.\cite{voinescu2013lightweight}


\subsection{The SparrowV3.2}
 
 
The other member of The Sparrow Family, the SparrowV32 nodes are responsible of collecting data from the surrounding environment and sending it to the Sparrow Dongle. The collected data depends on the sensors attached to the SparrowV3.2. The standard implementation has a light, temperature and humidity sensor. Besides this information, it also sends the battery level.

The SparrowV32 can be modiffied by adding aditional sensors for a better understanding of the environment.
