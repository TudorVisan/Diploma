Unitatile functionale ale unei placi video sunt:

\begin{itemize}

\item Pixel Shader.
\item Vertex Shader.
\item Geometry Shader.
\item Tessellation Shader.

\end{itemize}

\subsection{Pixel Shader}

Pixel shader este o unitate care se ocupa de culoarea pe care o are un pixel al ecranului. Acesta
unitate aplica efecte peste culoare de baza, efecte precum blur, shadowing, edge detection. Din
acest motiv, un pixel shader trebuie sa stie contextul pixelului pentru a putea calcula culoarea
finala.

\subsection{Vertex Shader}

Vertex shader este o unitate care poate calcula pozitia unui vertex. Mai exact, scopul lui este sa
transforme coordonatele 3D ale unui vertex in spatiul 2D corespunzator al ecranului. Pentru a
realiza acest lucru, un shader vertex poate lucra cu proprietatii ale liniei cum ar fi culoarea,
pozitia si coordonatele texturii, dar aceasta unitate nu poate crea vertexi noi. Rezultatul unui
vertex shader poate fi trimis unui Geometry Shader sau unui engine de Rasterizare.

\subsection{Geometry Shader}

Geometry Shader, adaugat in DirectX10, spre deosebire de Vertex Shader, poate crea vertexi noi.
Acesta primeste ca intrare un poligon, si poate sa adauge puncte noi pentru a crea poligoane noi si
un nivel mai mare de detaliu. Rezultatul acestei unitati este trimis lui Pixel Shader
\cite{kirk2007nvidia}.

\subsection{Tessellation Shader}

Tessellation Shader, introdus in DirectX11, permite obiectelor aflate in apropierea camerei sa fie
alcatuite din mai multe poligoane fata de obiectele mai indepartate. Acest lucru este realizat
hardware, si doar de catre acesta unitate, astfel, impactul asupra performantei este mic comparativ
cu nivelul de detaliu generat.
