\subsection{Future work}

We proved that the sensor is low power, while continuously collection data. As a future project, we will design a lower power node, that could be power by battery for longer periods. The microcontroller will be Atmel SAM R21, which brings the following advantages compared to the existing one:

\begin{itemize}
    \item 32 bit architecture instead of 8 bit architecture. Faster execution of programs, which
        will allow for higher power efficiency per MHz.
    \item 256 kB of flash instead 128 kb of flash with 150k flash write/erase cycles endurance
        instead of 50k, and 16k eeprom emulation with 600k write/erase cycles endurance instead of 4k eeprom with 100k write cycles. This will allow more data to be saved more often and bigger programs to be used.
    \item 32 kB of sram instead of 16 kB. This will allow for higher complexity programs and algorithms.
    \item Higher operating frequency, 48 MHz instead 16 MHz, while maintaining the same power
        consumption of 5 mA. This will allow the cpu to process more data, instead of sending it
        to base station for processing. It will save power, because cpu processing will be cheaper
        then sending data using the RF, allowing for longer battery life and more nodes connected
        to the same base station.
    \item 70% smaller footprint, 32QFN instead of 64 QFN. This allows for smaller nodes that can be mounted more easily.
    \item Small improvement in transceiver's power efficiency and transmission power, 4 dB instead of 3.5 dB.
    \item 12 bit ADC instead of 10 bit ADC for improved sensor reading accuracy.
\end{itemize}

Besides the cpu, the accelerometer that will be used will be LIS2DH12, which has a much more lower power consumption of just 6uA for a 50Hz ODR.
