\subsection{Conclusion}
In conclusion, over the course of the experiment we have observed that the 
software implementation of ECB and CBC encryption algorithms can not be used in low 
power wireless sensor networks as an alternative to the hardware AES because it 
is by its nature a slower solution and it does not integrate well with the 
concept of low power sensors.

We have seen that using this implementation, the number of encryption 
and decryption operations which can be performed in the unit of time 
is significantly smaller than when using hardware AES, up to a total 
of 7 times slower, especially in the case of the decryption operation.

A second drawback would be that the software implementation requires 
additional memory to store the tables used by the ECB and CBC methods 
to generate keys. This can take up to 21Kb of flash memory on the controller.
In comparison, when using hardware AES, no such memory usage is required.

Going even further, the software implementation requires even more resources 
in order to perform both encryption and decryption operations on the same node. The 4Kb of 
RAM memory available on the node's controller is not sufficient to perform these tasks, one node
being able to perform only decryption or only encryption.

The last test has shown that with the proposed headers for security, the coordinator (or 
gateway) node can still perform well even when flooded with a great number of corrupt 
packages meant to perform a DDoS attack and prevent the network from processing 
proper data.

\subsection{Future Work}
In addition to the AES implementation, there are a series of other encryption algorithms 
and methods which have been presented at the beginning of this paper. Future research and 
analysis can be performed to verify if any of those other methods could perform better or on 
the same level as the hardware AES on the given experimental setup.

Furthermore, the overall energy efficiency of different implementations over the course of 
long periods of time can be studied. The goal would be to establish which encryption and 
security method is best suited for the SparroE wireless sensor nodes in order to ensure 
their operational lifetime is the maximum possible.
