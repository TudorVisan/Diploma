In recent years, wireless sensor networks have been used more and more often as a cheap and easy to maintain monitoring system.
Wireless sensors are adequate tools for monitoring various environments due to a series of characteristics such as: 
\begin{itemize}
  \item low power consumption which increases autonomy and helps reduce the node size (no need to attach large batteries)
  \item possibility to attach energy harvesting modules which further help to increase their autonomy
  \item ability to mount numerous sensor peripherals on a small surface
  \item require a low amount of outside intervention and maintenance
  \item can operate in harsh areas or places which would be hard to reach by humans
\end{itemize}

Since these sensors communicate via wireless networks, they are ideal tools to use in monitoring remote or otherwise hostile 
environments such as: underground caverns, ocean floors, volcanic mountain ranges, etc. Another field in which wireless sensors 
would represent a good monitoring solution is geodynamics. Using such a network, it could be possible to predict natural phenomena such 
as earthquakes, tsunamis, landslides and volcanic eruptions much faster and with increased precision regarding magnitude and the 
time of the event. Also, such a solution may prove to be cheaper and more flexible than existing installations deployed for performing 
these tasks.

Delving even deeper into the utility of such an application, we can take into account that whenever a phenomena amongst those mentioned 
earlier occurs, the areas which often suffer damages are cities and towns. A good use for a wireless sensor network would be to mount it 
around a city and on buildings inside the city situated in such danger zones. Thus, whenever an earthquake or other natural disaster may occur, 
authorities can respond faster and reduce the damage, both material and human lives.

In this paper, we propose a wireless sensor network solution for monitoring earthquakes. The nodes can be mounted on buildings and 
they will monitor the vibration of the building as well as various other parameters (air pressure, temperature, etc.). Once calibrated 
to the "normal" values of the parameters, whenever these parameters go over a threshold value, a system is notified that there is a possible 
risk of an earthquake happening. More so, this system can be used to determine if a building is
exposed to deterioration due to external factors such as proximity to construction site, roads frequented by heavy load trucks, etc.

In the Architecture section we shall present how the SparrowV4 sensor's hardware and software components are designed and interconnected.
Then, in the Experimental section we present the laboratory tests which were ran in order to determine the viability of these sensors.
The Results section shall analyze the data acquired during the tests and show that the SparrowV4 wireless sensor nodes are reliable. Finally, 
in the Conclusions and Future Work sections, we discuss how the sensors can be further improved with new hardware components.
