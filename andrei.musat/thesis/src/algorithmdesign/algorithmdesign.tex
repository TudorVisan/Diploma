%========================= Algorithm Design =========================
This chapter presents how the application is designed at a logical level and describes the 
main algorithms used by the application.

\subsection{Planarity testing}

The first algorithm used by the application is the planarity testing method known as "path addition". In 
order to properly present this algorithm, we must first introduce the necessary notions regarding graph planarity 
and methods of testing this property.

\subsubsection{Planarity property and criteria}

A given graph G=(V,E) is planar if it can be drawn on a plane and its edges never cross each other, i.e. they 
intersect only at their endpoints. This type of drawing is also known as a planar embedding of the graph.

In order to determine if a graph possesses the planarity property, a series of theorems and criteria have been 
stated over the years. Amongst the first of these criteria is a theorem published by the Polish mathematician 
Kazimierz Kuratowski in 1930. The theorem deals with subdivisions, i.e. graphs which result from inserting vertices 
into edges. It states that a planar graph shall not contain a subdivision of the forbidden graphs K5(the complete 
graph on five vertices) or K3,3(complete bipartite graph on six vertices, three of which connect to each of the other 
three, also known as the utility graph). It is formulated as follows:

A finite graph is planar if and only if it does not contain a subgraph that is a subdivision of K5 or K3,3 .

Another important theorem which deals with planarity was formulated by the German mathematician Klaus Wagner. 
This theorem takes into consideration graph minors instead of subdivisions. A graph H is called a minor of a given 
graph G if H is obtained by deleting vertexes and edges or by contracting edges. The Wagner theorem states that 
a finite graph is planar if and only if it does not have K5 or K3,3 as a minor.

While these theorems manage to correctly define the planarity problem in a mathematical way, they are not optimal 
criterions to use in practice. The main reason is efficiency; we would like the complexity of such an algorithm 
to be linear O(n). In practice, there exist other theorems and criteria which fit in the linear complexity. For 
example, given a finite, connected planar graph with \iv the number of vertices, \ie the number of edges and 
\if the number of faces (regions bounded by edges, including the outer, infinitely large region), the following 
hold true:

Theorem 1: If \iv ≥ 3 then \ie ≤ 3\iv − 6;
Theorem 2: If \iv ≥ 3 and there are no cycles of length 3, then \ie ≤ 2\iv − 4;
Theorem 3: v − e + f = 2 (Euler's formula).

Unfortunately, these theorems are only necessary conditions, not sufficient conditions. They can only be used to 
prove that a graph is not planar; they cannot prove that a graph is planar.

\subsubsection{Vertex addition method}

The edge addition algorithm is the result an intensive research started in the 1960s by Abraham Lempel, Shimon Even
and Cederbaum. They created an algorithm to determine whether a graph is planar or not and embed it in the plane in 
O(n2) time. Later on, Even and Tarjan developed a method to generate the st-numbering of a graph in linear O(n) time.

***describe st-numbering here***

Finally, Kellogg Booth and George Lueker created a data structure which represents the possible embeddings of a graph 
(or, in practice, induced subgraphs of a graph). Using this data structure and the previous methods, they managed to 
created a linear time algorithm to determine planarity and possible embeddings. We will next present the general steps 
of an implementation of the Vertex addition method as proposed by Norishige Chiba and Takao Nishizekii.

First, it computes the st-numbering of the given graph. It then creates a PQ-tree which contains only one P node, the 
source, and all the other nodes are leaves. Next, a loop is entered which, for every leaf node, shall perform two steps:

a) A reduction step which attempts to gather all matching leaves into a P node which respects the st-numbering. If this 
step fails, then the graph is not planar.

b) A vertex addition step, in which full nodes are replaced by a single P node and all the neighbours of that node 
numbered higher than itself are added as leaves.

Figures x to y below present the steps of this algorithm applied on a graph G(3,3).

****** Insert figures here ******


\subsection{Graph placement}

Graph placement refers to assigning a position (set of coordinates) to each node of the graph in the space where 
the graph has to be represented.

We saw in the previous chapter that algorithms which determine if a graph is planar can also generate its 
embedings.
