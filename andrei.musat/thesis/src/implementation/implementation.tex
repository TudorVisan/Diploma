In this chapter we shall discuss the programming techniques, language characteristics and APIs
used in implementing each system component and algorithm. We shall highlight the APIs used and the 
integration with the Eclipse platform, as well as how OOP techniques and patterns are used to implement 
some of the algorithm presented in chapter \ref{Algorithm Design}.

\section{Eclipse platform and Draw-2D graphics API}

The Eclipse platform is an open source, cross-platform IDE(Integrated Development Environment) environment 
developed and maintained mainly by a consortium which includes companies such as IBM, Red Hat and SuSE. 
It mainly targets Java developers, but it also supports the plug-in additions to extend its functionalities 
for various programming languages. It can also work with typesetting languages such as LaTeX and be integrated 
with revision control systems like CVS(Concurrent Versions System) and GIT.

Eclipse has become popular amongst both software and hardware developers. Its main advantages are: faster 
code navigation and visualisation, better code understanding through tracing, referencing and hierarchical 
display functionalities and resource usage management and monitorization. The platform is also able to 
handle makefile generation, compilation and, most importantly, integration with debugging and profiling tools.

This IDE enviroment has been chosen mainly because of it can operate cross-platform and it is open source. 
Moreover, the code does not have to be written and organized differently in order to function on different 
platforms, such as Windows or OS-X. The drawbacks are also relevant, since running any application requires 
the entire platform to load, which is significantly more taxing on resources than a command line application 
would be. Thus, the IDE can be considered suboptimal from this perspective if the user is not fully interested 
in all of its capabilities.

Since it is an open source project, the Eclipse platform offers developers various APIs to facilitate and 
encourage the community to extend or improve existing functionality or add new capabilities. One such 
API is the Draw-2D graphics library, which is an extension of the SWT library. This library is generally used 
to implement small widgets which handle the drawing of charts and diagrams.

By default, the API can render any object which the SWT library can handle. In addition, it adds a series of 
classes and methods which allow the user control over the position of each object, its contents and other 
graphical characteristics such as colour and style. We shall continue by enumerating and discussing some of 
the main Draw-2D classes used in the implementation.

\subsection{Draw 2D Point class}

Points represent the basic operating unit of the library. They do not refer to a graphical entity, but a 
spatial one. They are used to define the location (coordinates) of a figure or shape, designate an axis 
or segment or define the cornes of a polygon when placed in an ordered list.

The class exposes API for accesing and manipulating coordinates, as well as basic mathematical computations 
such as transposing points and calculating the euclidian distance between two points. Still, the API lacked more 
advanced computations such as: manhattan distance, determining the angle between the X axis and the line 
designated by two poins and moving the point in a general direction with a given speed. These functions 
have been implemented and are shown below:

***************** aici bag cu functiile pe care le-am pus in biblioteca *****************

\subsection{Draw 2D Rectangle class}


