Core i7 a fost o arhitectura revolutionara, o arhitectura care odata cu performanta marita a ei a
adus si un consum mai redus de energie. Eficienta energetica a procesorului a permis ca intel sa o
poata folosi atat pentru sisteme mobile, de exemplu laptopuri, cat si pentru sisteme de calcul
ultra performante.

