\normalfont\normalsize
\chapter{Software Environment: Contiki Operating System}

The Contiki Operating System is an open-source, highly portable multi-tasking operating system for memory-efficient
networked embedded systems and wireless sensor networks. It is our choice at the moment as platform for this study. 
It offers IP communication, both IPv4 and IPv6. Currently the IPv6 implementation is in IPv6 Ready Phase 1, which 
means that some functionality related to neighbour discovery is not complete, leading to a lack of multi-hop routing.
Platforms supported by this operating system range from MSP430 from Texas Instruments and AVR Atmegas to old home computers.

The innovations introduced by this system were the introduction of IP communication in low-power wireless sensor networks,
with the uIP stack and the protothreads system, introduced in 2005. The best feature is the implementation of a network 
interface on the Raven USB Stick, such that nodes can be communicated with transparently from outside the wireless network.  

Contiki offers a collaborative multi-tasking environment, based on a coroutine mechanism, in which each process must at one
point yield the processor so that a new process is scheduled. Contiki has some hardware abstraction, especially for commonly used
part. Network communication is abstractized with a ``protosocket'', and libraries exist for energy estimation, timers, event timers, etc.

\clearpage

\section{Process kernel}
\subsection{Protothreads}

Protothreads are an innovation made for the Contiki system, although the system is based on the coroutines implementation in C,
which have been around for quite some time. In the Contiki operating system, they are called \textit{local continuations}, allowing
a subroutine (a function) to pause and resume execution of its code. These local continuations are formed with normal function that
call some defined macros.

\lstset{numbers=left, mathescape=true, nolol=false,caption=Coroutines in C (Simon Tatham),label=lst:cr}
\begin{lstlisting}
int function(void) {
    static int i, state = 0;
    switch (state) {
        case 0: /* start of function */
        for (i = 0; i < 10; i++) {
            /* so we will come back to "case 1" */
            state = 1; 
            return i;
            /* resume control straight after the return */
            case 1:; 
        }
    }
}
\end{lstlisting}

The code in \ref{lst:cr} demonstrates the use of coroutines, the repeated call of function() will yield results from 0 to 9. First 
it will change internal state from 0 to 1, posting a new label for the switch statement to jump to. Based on this example, one can 
easily define suggestive macros to generalize:


\lstset{numbers=left, mathescape=true, nolol=false,caption=Coroutine macros in C (Simon Tatham),label=lst:crmacro}
\begin{lstlisting}
#define crBegin static int state=0; switch(state)\
                       { case 0:
#define crReturn(i,x) do { state=i; return x;\ 
                      case i:; } while (0)
#define crFinish }
\end{lstlisting}

Indeed, the local continuations in Contiki are similar:

\lstset{numbers=left, mathescape=true, nolol=false,caption=Local Continuations,label=lst:lc}
\begin{lstlisting}
typedef unsigned short lc_t;

#define LC_INIT(s) s = 0;
#define LC_RESUME(s) switch(s) { case 0:
#define LC_SET(s) s = __LINE__; case __LINE__:
#define LC_END(s) }
\end{lstlisting}

The protosocket library build on top of this, adding yield functionality and the possibility to ``block'' waiting for a condition.
To do this each protothread holds an \texttt{lc\_t} structure together with a semantic of how it was interrupted (\texttt{PT\_WAITING},
\texttt{PT\_YIELDED}, \texttt{PT\_EXITED}, \texttt{PT\_ENDED}). Yield means giving up the processor, waiting is yielding until the 
protothread is scheduled and a condition is met, exited resets the state of the protothread (of the local continuation on which it is
based), end means the protothread has reached the end of the function and the state is reset (not too different from exit).

\subsection{Processes}

Processes are built upon the protothread structure, each process having one protothread. The functionality added to protothread is of 
context and higher-level abstraction. Processes in Contiki run as a result of responses to \textit{events}. An event can be triggered
by a timer, by the uIP stack (network event) or even by the process itself (a continue event used by a pause function in the process
library). 

\lstset{numbers=none, mathescape=true, nolol=false,caption=Process definition,label=lst:process}
\begin{lstlisting}
#define PROCESS(name, strname)                          \
  PROCESS_THREAD(name, ev, data);                       \
  struct process name = { NULL, strname,                \
  process_thread_##name }
\end{lstlisting}

The scheduler is a simple queue on which events are posted. When an event is popped, the associated process is called with the specific event.
The process is therefore a complex piece of code that retains state (through local continuations) and reacts on new input (through event data).
This adds versatility to the already existent protothread library.

Commonly, a process will have after declaration a specific thread function, its definition must begin with the macro \texttt{PROCESS\_THREAD},
it needs \texttt{PROCESS\_BEGIN} and \texttt{PROCESS\_END} macros and can then use wait, yield, pause macros for the control flow of the 
operating system (the responsiveness of the entire system depends on each process running, latency can very quickly become a problem because
of a slow process).


\section{Tasks in the Contiki context}
\label{sec:contikitask}
There are two possibilities regarding the implementation of generic tasks, one is to have them
all run under a single process, using timer events and with a scheduler similar to real-time operating
systems. A timer would be set to expire when the first task would be due. The disadvantage to this
method is a difficulty in coming up with a metric for processor use (versus idling).

The other possibility is to treat a task as a process. There would be a task manager that would be
able to start and stop tasks, alter settings, append subscribers, etc. A performance metric can be
calculated using this method concerning the scheduler. One of the tasks would measure the timeframe
between two succesive schedules of the same task.
As the system keeps more and more tasks running (as opposed to “waiting”, when they are still
scheduled), and more and more tasks use more CPU time, time between schedules will increase.

Starting the task would be similar to starting a process in Contiki, while stopping a task means
marking it as a WAITING process. While it is marked thus, in its (presumed) main infinite loop it
would wait until it is taken out of this state.

Obtaining information from the task can be done in two ways:
\begin{itemize}
 \item Collecting data directly, with get/set <parameter>
 \item Data sink method, where the entity that connects to the sensor can register itself or another
entity as a data sink for the output that is generated from the task. Parameters can be given such as the frequency 
with which data is forwarded to the subscriber (division of the frequency with which data is generated
\end{itemize}

\lstset{numbers=none, mathescape=true, nolol=false,caption=Task definition,label=lst:task}
\begin{lstlisting}
#define TASK(title, process)\
PROCESS(process, title);\
static struct task_list_t el_##process ={\
        .value={\
                .name=title,\
                .status=WAITING,\
                .app=&process}\
}
\end{lstlisting}

A list of processes can be kept with the linked list API available in Contiki. The structure that represents the task would be:

\lstset{numbers=none, mathescape=true, nolol=false,caption=Task structure definition,label=lst:taskstruct}
\begin{lstlisting}
typedef struct
{
        struct process * app;
        uint8_t status;
        char name[10];
        subscriber_list_t subscribers;
} task_t;
\end{lstlisting}

\section{Energest - Energy Estimation Module}

Energest is at the base of the \textit{Communications Power Accounting} power in the Contiki Operating system. The aim is to 
present proportional data regarding the types of wireless transmission taking place, with the scope of finally determining
which component uses more power. Energest supports multiple levels of accounting for energy tasks, and has a counter for each.
Time is measured with a real time clock between checkpoints and it is added to the counter of the active levels. This is how
the communications power accounting measures transmit and listen times, and will be useful for MAC protocols that have power
management. However, the stack on the AVR does not support this, so we will opt for a hardware alternative of measuring power,
as shown in \ref{sec:metrics}.

