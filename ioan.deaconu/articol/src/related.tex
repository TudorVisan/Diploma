Research has been done in the area of wireless sensor networks with mobile
gateways, but none include the same constraints as in our networks. Where most
research focuses on routing packets to a mobile sink, we consider applications
where network lifetime and low-maintenance are more important, and as such
energy consumption on the nodes must be kept to a minimum (in order to have 3+
years of network lifetime with no maintenance/battery changes). Shifting the
energy consumption to the sink is a direct consequence of this decision, under
the assumption that there exists a flight path that allows the gateway to
collect information from all the nodes. 

Two-Tier Data Dissemination\cite{ttdd}, Line-Based Dissemination\cite{lbd} are
protocols with mobile sinks for very specific networks. TTDD assumes sensor
platforms are arranged in a perfect grid pattern, while LBD establishes a
communication pattern over a virtual infrastructure, derived from the position
of the nodes. Nodes in both protocols do not have duty-cycles in their activity
and have either fixed-pattern positions or have GPS or equivalent equipped. This
is very different from the small, energy constrained sensor platforms we wish to
use and the mechanisms used by these protocols are not applicable in our system.

Another type of mobile sink system involves proactive routing as a variation to
the directed diffusion protocol\cite{dirduf}, such as XYLS\cite{xyls}, which
disseminates data reports in two geographically-opposite directions, such that
it intersects a corresponding data query. This is once again a system which
offers a query system to a mobile sink, but cannot offer the collection of all
measured data since its last fly-by.
