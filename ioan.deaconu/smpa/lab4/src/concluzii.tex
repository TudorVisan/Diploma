Desi intial au fost gandite ca accesorii pentru pasionatii de jocuri, odata cu trecerea timpului si
cresterea nevoii de a avea o putere de calcul cat mai mare, placile video au evoluat  de la simple
adaptoare grafice, care putea sa afiseze o imagine pe ecran, pana la acceleratoare de calcul.
Aceste noi tehnologii prezente in GPU-urile actuale permit accelerarea calculelor, astfel ca
programe de genul Matlab, programe de randare grafica, unde trebuie calculata interactiunea
fiecarei raze de lumina cu fiecare obiect, si chiar simulari de fizica a obiectelor sa fie
calculate in timp de real. Consumul lor este in continua scadere, ceea ce a permis placilor grafice
mobile sa fie si ele din ce in ce mai rapide, scopul celor de la Nvidia fiind ca intr-un viitor
apropiat, sa nu mai existe nici o diferenta de performanta intre placile mobile si cele desktop. 

Desi sunt de sute de ori mai rapide ca un procesor in aceste programe,
placile video inca nu sunt suficient de avansate incat sa poata rula independent, ele find
specializate pe calcule matematice si grafica video.



