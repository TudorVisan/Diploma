\chapter{Conclusions}

This study has shown that scheduling on Wireless Sensor Networks can be approached as a graph-cut problem with success,
because graph algorithms are easy to implement and at the same time powerful and versatile. Its versatility  was clear 
in \ref{sec:adapt}, where new constraints related to energy efficiency were easily integrated into the algorithm.

The main contribution of this work was defining the problem of task scheduling as a more accurate reflection of reality, 
together with considering this a clustering / minimal cost partition (min k-cut) problem. The model was not destined to 
be simulated, it was based on real hardware platforms and inputs of the scheduler are consistent with the real state of
the network in a moment of time. This makes the implementation of the algorithm easy and consistent with the theory. 

Even though the asymptotical complexity is quite large in the case of this algorithm, the static case can be expected 
to appear more often, with time for scheduling not being a constraint. The main constraint in wireless sensor networks 
is energy, and scheduling the tasks in a network can finish before the network is enabled. Aproximation methods
exist and their running times are a lot smaller for at most twice the energy spent.

With rising interests in smart interfaces, services and self-organizing networks, dynamic scheduling is increasing in 
importance. We've outweighed two directions of optimization for greedy schedulers, one to maximize network lifetime and
one to minimize energy spent communicating (which are not the same criteria). 

In the final part we have shown how an application for wireless sensor networks can be broken down into tasks and how
to represent their hardware requirements (we used the term \textit{node affinity}. Tasks can be of a few varieties,
sensing tasks (personalized for each sensor or each node - hardware specific), event detection tasks, actuating tasks
(the reverse of sensing tasks) and generic processing tasks. Along with these we have added \textit{metatasks} which 
aid in the process of scheduling and network status monitoring.

\section{Outlook}

There are two issues that could not be discussed as in depth as it was necessary. One is the influence of multi-hop 
communication on energy consumption, that we have not accounted for and the other is the number of tasks that is optimum
for a given topology and data dependencies.

These issues are closely related and could not be explored due to lack of multi-hop routing in our software environment,
which will be available some time after this study is finished.

Multi-hop communication would significantly alter the formalisation of the problem. Energy would no longer be spent only 
by the transmitter and the receiver, but by any node that forwards the packets between them, with double the energy 
(receive / transmit). The scheduler would need to be fully-aware of the topology and calculate the cost of each data 
dependency proportional to the number of hops needed to reach the destination.

The second issue borders self-organisation ideas for wireless sensor networks, but instead of choosing a local leader, the scheduler
chooses how many times to assign a task. This would lead to the tasks that gather data from other sensors and process them
to be closer to the source of the data then before.

There is a method to implement this last idea within a single-hop network, but it would not be as effective as in a multi-hop one.
The idea resembles the one for tasks that have to run on all capable nodes. Because the algorithms makes steps of 2-cuts, we 
will discuss that case: Let us suppose that we have already obtained the minimum s,t-cut, with the tasks with variable number 
of instances all in the sink set. We would assign a copy to the source set only if the sum of the edges going to and from 
nodes outside the source set are smaller than the sum of edges going to and from the nodes inside the source set.