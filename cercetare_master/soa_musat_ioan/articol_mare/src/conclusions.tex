\subsection{Conclusions}

In this paper, through the previously presented experiment, we have shown 
that the Sparrow v4 wireless sensor node is a viable tool for monitoring 
earthquake activity. It is also versatile as it can fulfil multiple purposes 
and tasks. Sparrow v4 nodes can be placed in key points on a structure to monitor 
the amount of vibrations which it is exposed to. Alternatively, the nodes can 
be organized in a wireless sensor network and can be used to monitor seismic activity 
in certain areas. They can act as a tool which is used to predict earthquakes and 
warn authorities in case of danger.

The results have shown that the experimental rig is responding similar to a tall building because 
the lower positioned nodes gather lower amplitude values than the ones located on higher shelves 
when the table is vibrating. At the end of the simulation, when the table is no longer being shaken 
and it is naturally vibrating due to previous forces, the higher positioned nodes continue 
to detect vibrations, as it is most likely to happen in the case of a real building. Even more, the 
system could detect flaws in various structures and prevent real earthquakes from causing their collapse.

Furthermore, because the sensor nodes are low power, no matter in which type of application they 
are used, they will be able to function for prolonged periods of time, which makes them ideal 
for monitoring environments.

\subsection{Future Work}
The Sparrow v4 sensor nodes can be further improved by adding energy harvesting modules to their 
current design. Allowing them to harvest resources such as solar power can further increase 
their autonomy and efficiency. Further more, there is the possibility of designing a node which needs even less power to function. 
This can be achieved mainly by swapping the microcontroller with an Atmel SAM R21, which brings the following advantages compared to the existing one:

\begin{itemize}
    \item 32 bit architecture instead of 8 bit architecture.
    \item 256 kB of flash instead 128 kb of flash. 150k flash endurance instead of 50k , and 16k eeprom emulation with 600k write cycles endurance instead of 4k eeprom with 100k write cycles. This will allow more data to be saved more often and bigger programs to be used.
    \item 32 kb of sram instead of. This will allow for higher complexity programs and algorithms.
    \item Higher operating frequency, 48 MHz instead 16 MHz, while maintaining the same power consumption of 5 mA.
    \item 70 percent smaller footprint, 32 QFN instead of 64 QFN. This allows for smaller nodes that can be mounted more easily.
    \item Small improvement in transceiver's power efficiency and transmission power, 4 dB instead of 3.5 dB.
    \item 12 bit ADC instead of 10 bit ADC for improved sensor reading accuracy.
\end{itemize}

Another component which can be swapped is the accelerometer. There are models which have a consumption of only 6uA for a 50Hz ODR.
Howver, the trade-off for these changes would be a sight increase in the cost of the nodes.
