Wireless Sensors are low cost, low power devices optimized to perform custom tasks. They usually
gather information from their surroundings and send it to a base station servers in order
to be stored and processed. This communication is generally achieved using gateways. A
gateway is usually connected to a more powerful device that can process the received information and take certain actions based on the results. 
The information transmitted by wireless sensors often represents sensitive data. For this reason, security protocols are implemented to 
prevent attacks that can intercept, replicate or alter the data.

Currently, security protocols in wireless sensor networks rely mostly on key based encryption algorithms. While this method can achieve 
great efficiency in terms of data security and protection, it also requires a high amount of computational power and is not 
always a task which is swiftly executed. More so, in order to use such protocols, nodes must store all the necessary keys.
Due to their design, wireless sensors often do not possess the required resources. They seldom have external memories attached 
to them and their processing power is limited to microprocessors which run at frequencies in the range of 1-20 MHz.

Another limitation of using this type of protocols to encrypt data is related to energy consumption. Usually, these sensors are 
powered by small batteries with a limited capacity. If the microprocessor has to perform intensive computations, these batteries 
will be depleted in short amounts of time. Even equipping sensors with energy harvesting peripherals does not ensure that 
the battery lifespan is significantly increased.

This paper analyzes whether software AES is a viable low power solution for sensors which may not have access to hardware AES modules 
or which do not wish to rely on the supplied AES algorithms. The analyzed encryption methods are ECB and CBC, the same which are 
available on the SparrowE's Zigbee 900 transceiver.

The analysis answers the question of whether a balance point between security and energy consumption can be achieved for the software AES approach. 
Should this method be viable, it can be implemented on cheaper models of Sparrow wireless sensors or even be used as a secondary layer of data security 
in such a network.
