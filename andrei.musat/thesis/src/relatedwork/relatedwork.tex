Even though graph embedding and diagram drawing is a popular subject which has been the topic of articles and research, 
actual open source software which can do such operations is scarce. Algorithms which solve embedding problems have been 
described and implemented since the year 1970. The major impediment for any application which would like to perform 
diagram drawing is actually the diversity and ambiguity of the problem. Apart from performance analysis (time and 
memory consumption), and corectness of the drawing, any other aesthetic appreciation of the result is subjective.
Depending on who the diagram is addressing and which its final scope is, certain styles, connections and layouts are 
appropriate, while others render the final result unasable.

This does not mean that a middle-ground does not exist and there are no techniques or conventions which are 
generally accepted as good practice. There two open source libraries of diagram drawers: Graphviz dot and 
Eclipse GEF. Mai multe despre ele maine cand sper sa termin bullshit-ul.

\subsection{Graph Placing}

The problem of finding the optimal placement for the nodes of a graph in the plane
in which it will be drawn is a well known graph theory/computer science problem
which has been tackled by many researchers.

The algorithms which are currently used most frequently by available software are:
grid placement , tree placement (in cases where it is possible), and force-directed placement \cite{hu2005efficient}.

Grid type algorithms ensure that the components of a graph are placed on hierarchical layers within a
grid structure. Certain optimizations, such as placing the figure with most connected edges in the middle,
are used to ensure more clarity. However, while the algorithm has proven to be fast at placing graphs,
the resulting connections often overlap and make it hard to determine where they go or what they represent.

Tree algorithms work only on a specific class of algorithms. If the given graph is not a tree or cannot be
converted to a tree, the algorithm cannot be applied. The results, though, are a clear and easy to follow
drawing of the graph, with good performance.

The force-directed placement method is a relatively newer and a better approach than the other algorithms mentioned.
It works by modeling the graph as a physical system in which the bodies (nodes) interact with each other via
forces (edges which make up connections). There are two sets of forces which affect nodes: those that attract
them to each other (e.g. elastic force) and those that repel them (e.g. electromagnetic forces). By using
this model, nodes directly connected to each other will be kept together and form clusters while unrelated
nodes will be repelled towards the outer margin of the graph. While the results of this specific method are
impressive visually, performance wise it is not exactly optimal because it requires successive iterations
until a stable configuration has been reached.

\subsection{Connection Routing}

Regarding connection algorithms, most software today either use techniques such as shortest path connection, orthogonal edge routing or various routing mechanics built around heuristics. Each approach tries to find a 
balance between the clarity of the representation (number of edges in a connection, length of a connection, 
number of intersections etc.) and the area needed to draw the connections. Ideally, each connection should be 
relatively short, in a direct path with as few intersections with other lines as possible. However, the problem
of finding the proper representation so that the number of intersections is 0 is NP , therefore there is no
general solution, only approximations. Currently available software generally allow certain intersections or 
overlaps of connections to happen in order to compute faster and not lose performance.
