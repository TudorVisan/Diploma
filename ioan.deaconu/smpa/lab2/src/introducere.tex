Intel, pe langa AMD, este unul din liderii mondiali ai industriei de microprocesoare. Acest lucru
este datorat in mare parte arhitecturii Core i7, arhitectura revolutionara lansata in noiembrie
2008.

Desi de atunci Intel a ajuns la a 5-ea iteratie a arhitecturii, sub nume de cod Broadwell, functionarea
 ei este identica cu cea a arhiecturii Nehalem, prima iteratie a lui Core i7.

Din aceast motiv, aceasta lucrare va prezenta imbunatatirile aduse acestei arhitecturi, imbunatari care
chiar si dupa 7 ani de la introducere lor sunt inca folosite.

