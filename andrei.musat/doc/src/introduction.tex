Diagrams are a tool which is widely used in various domains in order to represent
various concepts, hierarchies, architectures etc. However, there is a field in 
which diagrams are a key part of development: FPGA design. Here, diagrams are used
to describe blocks \cite{jeon2007verification}, to represent the way the modules of a chip are connected at a 
logical level, to show how multiple micro-chips are connected on a board. In FPGA 
and ASIC verification, they are used to represent class hierarchies and inheritence, 
agents, sequencers and general test flows.

Even though they are an important part of the FPGA design and verification world, 
diagrams do not receive the required attention from the developer community. Despite 
their utility in understanding and maintaining the coherence and logical flow of a
design, they are not always taken into consideration by developers of IDEs and other 
design and verification solutions. Usually, an FPGA design consists of tens, maybe hundreds 
of thousands of design elements; the equivalent of a graph with the same number of nodes. 
Representing such a design as a diagram would be, in turn, equivalent to drawing said graph 
in a plane \cite{de1990draw}. For current algorithms and software, this is a well known problem which is 
difficult to solve in a decent amount of time and also yield a pleasant and easy to 
understand result. We aim to show in this paper that there exists a solution for graph 
and diagram layouts that is more versatile, uses newer technologies and programming 
techniques and gives results which feel more natural for the user.

We introduce a layout and routing solution which uses genetic algorithms and orthogonal
routing \cite{wybrow2010orthogonal}. We will show an overview of the system architecture in Chapter \ref{chap:arch}, 
the hardware and software implementation for the two algorithms in Chapter \ref{chap:impl} 
and results of using the algorithm in Chapter \ref{chap:results}. 
