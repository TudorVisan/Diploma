\chapter{Contiki API}
\section{Process macros}
\begin{itemize}
\item \texttt{PROCESS\_THREAD (name, ev, data )} - Define the body of a process.
This macro is used to define the body (protothread) of a process. 
The process is called whenever an event occurs in the system, 
A process always starts with the PROCESS\_BEGIN() macro and end with the PROCESS\_END() macro. 
\item \texttt{PROCESS\_BEGIN ()} - Define the beginning of a process. 
\item \texttt{PROCESS\_END ()} - Define the end of a process. 
\item \texttt{PROCESS\_YIELD ()} - Yields the currently running process
\item \texttt{PROCESS\_WAIT\_EVENT\_UNTIL (c)} - Wait for an event to be posted to the process, with an extra condition.
This macro is very similar to PROCESS\_WAIT\_EVENT() in that it blocks the currently running process until the process receives 
an event. But PROCESS\_WAIT\_EVENT\_UNTIL() takes an extra condition which must be true for the process to continue.
\item \texttt{PROCESS\_PAUSE} - Yield the process for a short while.
This macro yields the currently running process for a short while, thus letting other processes run before the process continues. 
\end{itemize}

\section{uIP functions}
\begin{itemize}
 \item \texttt{PSOCK\_INIT (psock, buffer, buffersize)} -  Initializes a proto-socket.
This macro initializes a protosocket and must be called before the protosocket is used. The initialization also 
specifies the input buffer for the protosocket.
\item \texttt{PSOCK\_SEND (psock, data, datalen)} - Send data.
This macro sends data over a protosocket. The protosocket protothread blocks until all data has been 
sent and is known to have been received by the remote end of the TCP connection.
\item \texttt{PSOCK\_READBUF (psock)} - Read data until the buffer is full.
This macro will block waiting for data and read the data into the input buffer specified with the call to 
PSOCK\_INIT(). Data is read until the buffer is full..
\item \texttt{CCIF process\_event\_t tcpip\_event} - The uIP event.
This event is posted to a process whenever a uIP event has occurred. 
\item \texttt{CCIF void tcp\_listen (u16\_t port)} - Open a TCP port.
This function opens a TCP port for listening. When a TCP connection request occurs for the port, 
the process will be sent a tcpip\_event with the new connection request.
\item \texttt{struct uip\_conn *tcp\_connect(uipipaddr\_t *ripaddr,u16 port, 
void *appstate)} - This function opens a TCP connection to the specified port at the host specified with an IP address. 
Additionally, 
an opaque pointer can be attached to the connection. This pointer will be sent together with uIP events to the process.
\item \texttt{uip\_connected()} - Has the connection just been connected? 
\item \texttt{uip\_closed()} - Has the connection been closed by the other end? 
\item \texttt{uip\_aborted()} - Has the connection been aborted by the other end? 
\item \texttt{uip\_timedout()} - Has the connection timed out? 
\item \texttt{uip\_newdata()} - Is new incoming data available? 
\item \texttt{uip\_close()} -  Close the current connection. 
\end{itemize}

\chapter{Node capabilities}

\begin{table}[htb]
 \centering
\begin{tabular}{|l|c|c|c|}
\hline
Task & AVR Raven\texttrademark & Sparrow & Sparrow Power\\
\hline
Temperature sensing & \tick & \tick & \\
\hline
Humidity sensing & & \tick & \\
\hline
Voltage \& Current sensing & & &\tick  \\
\hline
Event detection & \tick & \tick & \tick \\
\hline
Alarm beep & \tick & & \\
\hline
LED signal & \tick & \tick & \\
\hline
\end{tabular}
\caption{Node capabilities}
\label{tab:capabilities}

\end{table}