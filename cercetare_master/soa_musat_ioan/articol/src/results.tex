In this chapter we present the experimental setup which was 
used to test the transmition of encrypted data, as well as results 
related to the power efficiency of the sensors, the processing 
capabilities of the proposed solution and resilience towards 
flooding and DDoS attacks.

\subsection{Experimental setup}

The above mentioned tests were performed using a small network 
composed of 3 SparrowE wireless sensor nodes. In this setup, 
one of the SparrowE nodes was designated as the network coordinator 
while the other two nodes were plain sensors, as it can be seen in 
the figure below:

\begin{figure}[ht] \centering
  \includegraphics[width=0.5\textwidth]{img/experimental-setup.png}
  \caption{The SparrowE WSN experimental setup}
\end{figure}

The network coordinator is directly connected to the base station and 
its functions include analyzing incoming packets, decrypting the valid 
ones and passing the information on to the base station. The rest of the 
nodes are simply sensors which encapsulate and encrypt the data from their 
accelerometers and then broadcast it on the network.

\subsection{Software ECB and CBC efficiency}

The metric of interest is the number of encryption and decryption operations 
on a 16 bytes block which can be performed in a unit of time by the software 
implementation of the ECB and CBC algorithms and if it scales linearly like 
the AES implementation does.

In order to gather the required data, both EBC and CBC have been performed on 
data from the node's sensors for periods of time lasting 1, 2, 3, 4 and 5 seconds 
respectively. Then, the same was done using the AES implementations. The results 
can be observed in the tables shown below:\\

\captionof{table}{Number of operations performed by software/hardware ECB}
\begin{tabular}{ | l | p{1.5cm} | p{1.5cm} | p{1.5cm} | p{1.5cm} |}
    \hline
    Time & ECB Software Encryption & ECB Hardware Encryption & ECB Software Decryption & ECB Hardware Decryption \\ \hline
    1 & 901 & 3552 & 520 & 3550 \\ \hline
    2 & 1805 & 7105 & 1044 & 7101 \\ \hline
    3 & 2707 & 10661 & 1562 & 10653\\ \hline
    4 & 3604 & 14221 & 2084 & 14205\\ \hline
    5 & 4508 & 17761 & 2603 & 17752\\ \hline
\end{tabular}\\\\

\captionof{table}{number of operations performed by software/hardware cbc}
\begin{tabular}{ | l | p{1.5cm} | p{1.5cm} | p{1.5cm} | p{1.5cm} |}
    \hline
    time & CBC software encryption & CBC hardware encryption & CBC software decryption & CBC hardware decryption \\ \hline
    1 & 873 & 3550 & 509 & 3380 \\ \hline
    2 & 1747 & 7103 & 1023 & 6760 \\ \hline
    3 & 2623 & 10651 & 1530 & 10143 \\ \hline
    4 & 3495 & 14203 & 2038 & 13525 \\ \hline
    5 & 4367 & 17754 & 2546 & 16901\\ \hline
\end{tabular}\\\\

As it is seen in both tables, the results show that the software implementation also 
scales linearly, just like the hardware AES. Unfortunately, the software AES proves to 
be almost 7 times slower than the hardware AES. This is especially problematic in the 
case of data decryption, where the software AES will be much slower will take longer 
periods of time to detect possible corrupt packets. Furthermore, this will also cause 
the data from the sensors to be gathered less often and there is the danger that 
an interruption which occurs when sensor data is ready could suspend the decryption 
process.

One more issue encountered during the tests shows that due to the limited amount of 
RAM memory available on the controller, only 8Kb, a node cannot perform both software 
encryption and decryption at the same time.

\subsection{Power consumption}

Under normal circumstances, the Atmega controller on the SparrowE nodes functions at a 
supply voltage of 3.3V and a frequency of 8MHz. If the nodes run the encryption by using 
only the software method, then the transceiver shall mostly be turned off while the sensor is 
running and its power consumption is negligible.\\


\captionof{table}{power consumption of a single ECB operation}
\begin{tabular}{ | p{1.75cm} | p{1.75cm} | p{1.75cm} | p{1.75cm} |}
    \hline
    ECB Software Encryption & ECB Hardware Encryption & ECB Software Decryption & ECB Hardware Decryption \\ \hline
    0.0358 J & 0.00957 J & 0.062 J & 0.01 J \\ \hline
\end{tabular}\\\\


\captionof{table}{power consumption of a single CBC operation}
\begin{tabular}{ | p{1.75cm} | p{1.75cm} | p{1.75cm} | p{1.75cm} |}
    \hline
    CBC software encryption & CBC hardware encryption & CBC software decryption & CBC hardware decryption \\ \hline
    0.0369 J & 0.00958 J & 0.0639 J & 0.01 J \\ \hline
\end{tabular}\\\\



While performing encryption operations, it has been measured that the controller draws an average current 
of 9.78 mA . If the hardware approach is used and the transceiver is turned on for large periods of
time, the measured current increases by an average of 610uA.

Because software AES operations are performed slower than their hardware AES counterparts, 
it would seem that the total power consumption of the sensor nodes would be between 4 and 7 times greater.
However, when calculated together with the power consumption of all the other components, 
results show that using software AES decreases the sensor's autonomy by a factor of only 
2 or 3. 

\subsection{Flooding and DDoS protection}

Inside the wireless network running at 900MHz, the nodes communicate by broadcasting packets.
The network coordinator node shall receive and analyze any packet which arrives on this channel.
This opens up the risk of exposing the network to a DDoS attack. In order to determine how 
resilient the network is to such an attack and how much useful information the coordinator 
loses because it spends time analyzing and decrypting corrupt packets, such an attack has been 
simulated inside the experimental network.

The simulation is done by expanding the network to 5 SparrowE nodes and having some of them transmit 
corrupt packets which are only meant to flood the network. The total size of these packets is 
112 bytes. The size must be a multiple of 16 in order to be processed by the block encryption 
algorithms. Then, these packets will be sent at time intervals starting at the network's default 
250ms and scaling down until the coordinator no longer receives any packets. The total number 
of packets sent during each iteration is 1500. The packet header is described below:

\begin{lstlisting}
#define MAX_PACKET_DATA 17

typedef struct __attribute__((packed)){
	int16_t x;
	int16_t y;
	int16_t z;
} data_t;

typedef struct __attribute__((packed)){
	uint16_t node_id;
	uint32_t timestamp;
	uint16_t frame_index;
	uint16_t nr;
	data_t data[MAX_PACKET_DATA];
} frame_t;
\end{lstlisting}

The results can be seen in the figure below:

\begin{figure}[ht] \centering
  \includegraphics[width=0.5\textwidth]{img/packet-loss.png}
  \caption{Graphic representation of packet loss correlated with interval between packet sending}
\end{figure}

We can see that packet loss does not drastically increase in the interval of 250ms - 40ms. However, 
once we start sending packets once every 32ms or faster, the network becomes more and more flooded.
When the transmission interval is below 27ms, the corrupt packets no longer reach the coordinator
node. This is due to a limitation of the SparrowE firmware which blocks the node when the transmition 
interval is too small. The overall packet loss for the 5 sensor network transmitting at 
intervals of 250 milliseconds is 2 percent.
