% vim: set tw=78 sts=2 sw=2 ts=8 aw et ai:
\documentclass[conference]{IEEEtran}

\usepackage{ucs}
\usepackage{url}
\usepackage[utf8x]{inputenc}
\usepackage[english]{babel}
%\usepackage{hyperref}	  % use \url{http://$URL} or \href{http://$URL}{Name}
\usepackage{underscore}	  % underscores need not be escaped
\usepackage{subfigure}
\usepackage{verbatim}
\usepackage{float}
\usepackage{amsmath, amsthm, amssymb}
\usepackage{parskip}
\usepackage{flushend}

% Support for including graphics
\usepackage{graphicx}
\DeclareGraphicsExtensions{.pdf,.png,.jpg}

\begin{document}


\title{Geodynamics monitorization using wireless sensor networks}

\author{
\IEEEauthorblockN{Ioan Deaconu, Andrei-Alexandru Musat, Dan Tudose}
\IEEEauthorblockA{
  Automatic Control and Computers Faculty\\
  University Politehnica of Bucharest,\\
  \{ioan.deaconu@cti.pub.ro\}, \{andrei.musat@cti.pub.ro\}}, \{dan.tudose@cs.pub.ro\}
}


 
\maketitle

\begin{abstract} 
% ********** Abstract **********
\chapter*{Abstract}

Descriere de maxim o pagină a lucrării în termeni cât mai generali (motivație, ce rezolvă, etc)

\textbf{Keywords} cuvinte cheie

\chapter*{Acknowledgements}

Mulțumiri 



% ********** End of Abstract **********

\end{abstract}

\begin{IEEEkeywords}
Wireless Sensor Networks, Geodynamics, Earthquake monitoring, Low power
\end{IEEEkeywords}

\section{Introduction}
\label{sec:introduction}
\normalfont\normalsize
\chapter{Introduction}

Technology has been present in our lives for a long time and we have become so acustomed to it that we do not even realize when we are using it.

This tehnological advance has lead to inventing small devices with low power consumption and communication capabilities. They form a wireless sensor network and can be used to collect data from the environment and send them to a gateway for further processing. In a way, even the new gadgets that have appeared this year, gadgets like smartwatches and fitness trackers, can be seen as wireless sensor devices because they collect data like stepts, heart rate and have wireless capabilities.

This increased usage of WSNs has created new posibilities of better understanding of our souroundings

The applications of a wireless sensor network are unlimited in number. For example they can be used to monitor crops, to detect possible forest fires, to detect the presence of animals or vehicles in certain areas, tracking and monitoring doctors in hospitals, guidance interactive toys, detecting and monitoring car thefts etc.


\section{Related Work}
\label{sec:related}
\normalfont\normalsize
\chapter{Related Work}

Related work 


\section{Architecture}
\label{sec:architecture}
\label{chap:arch}
\subsection{Hardware}

The processing power and wireless capability of the wireless sensor nodes are provided by an Atmel
ZigBit 900MHZ RF module. It contains an ATmega 1281V 8-bit microcontroller connected to an
AT86RF212 RF Tranceiver via a SPI interface. The Atmega 1281V is an low power 8 bit microcontroller
that is connected to the onboard sensors of the node. In order to be able to transmit or secure the
data, the microcontroller will communicate with the RF Transceiver. The Transceiver controller is a very low power chip,
capable of sending data up to 6 km. Also, the Transceiver contains a security module compatible
with AES-128. It supports hardware encryption and decryption for AES 128 ECB, but but for the AES
128 CBC it is available only the hardware encryption.


\begin{figure}[ht] \centering
  \includegraphics[width=0.5\textwidth]{img/wsn-soa-system-arch.png}
  \caption{System Architecture}
\end{figure}

\subsection{Software}

From the software perspective, the architecture is composed of two parts: gathering data from the sensors and encrypting it before transmitting it to another sensor of the central gateway.
The encryption method for the data shall use both of the hardware supported methods: AES 128 ECB
and AES 128 CBC. It is necessary to use them both because the nodes shall be transmitting 
two kinds of packets: a first type of packets which contains non-sensitive data and is used by the
receiver to identify the sender and a second type of packets which contains the actual sensitive data. Since it supports 
both encryption and decryption, the first set of packets, those containing identification information, shall be encrypted using ECB. The data itself shall be encrypted using CBC. Since CBC 
decryption is not supported by default, a software CBC decryption implementation is necessary in order to allow the sensors to verify the data at regular intervals.

 
\begin{figure}[ht] \centering
  \includegraphics[width=0.5\textwidth]{img/send-receive.png}
  \caption{Packet Types}
\end{figure}


In addition to the CBC decryption, it is also required to have implementations of other software encryption algorithms in order to perform the performance analysis. The chosen 
algorithms which shall be used in benchmarking are Skipjack and RC5.


\section{Experimental setup}
\label{sec:experimental}
In order to test the accuracy and capability of the Sparrow v4 wireless sensor nodes, we 
have built a shake table in order to simulate the movements of a building during an earthquake.
The purpose of the experiment is to see if the Sparrow v4 nodes can detect the main waves which 
are felt during an earthquake, P-waves and S-waves.

P-waves stands for Primary Waves and is a type of seismic wave produced when an earthquake occurs.
Amongst seismic waves, it has the highest velocity, making it the first wave which is recorded by 
conventional seismographs. This wave is formed by alternating compressions and rarefactions of the 
material and its mode of propagation is always longitudinal. The other type of wave which can be 
recorded during an earthquake is the S-wave which stands for secondary or shear wave. S-waves move 
in a transversal manner through the body of an object, thus resulting in motion which is perpendicular 
to the direction of wave propagation. 

The experimental setup tries to emulate the effects of these waves on a tall building which has Sparrow v4 
nodes mounted on its sides. The purpose of the experiment is to see how accurate the data harvested by the 
IMU is and how the amplitude of the movement varies from floor to floor.

\begin{figure}[ht] \centering
  \includegraphics[width=0.5\textwidth]{img/mounted-sparrow-nodes.jpg}
  \caption{SparrowE wireless sensor nodes mounted on the experimental rig}
\end{figure}

In order to perform the simulations, an experimental rig was built. This rig uses a movable frame as its base, 
and a tall, layered shelf mounted on top of it. The frame is composed of two separate planes which are moved 
individually by two DC motors. One plane produces left-right movement, while the other moves the shelf forward and 
backward. The planes are connected to the corresponding motor via a crank whose length can be adjusted, thus giving the 
ability to control the amplitude of the movement. Smaller amplitudes will result in sudden and violent shakes while 
higher amplitudes will allow the shelf to sway more. The schematic of the movable frame is presented below:

\begin{figure}[ht] \centering
  \includegraphics[width=0.5\textwidth]{img/experimental-rig.png}
  \caption{Experimental rig moving frame diagram}
\end{figure}

During the experiment, we harvest the accelerometer data from the Sparrow v4 sensors' IMU at various movement speeds.
The movement speed is controlled by the supply voltage on the motors. The supply values we chose for the experiment 
are as follows:
\begin{itemize}
  \item 0V representing a still table influenced only by the surrounding environment's vibrations
  \item 2V simulating a low impact vibration (earthquakes on the lower end of the Richter scale)
  \item 7V simulating a medium impact vibration (earthquakes on the middle of the Richter scale in the area where humans can feel them - over magnitude 4)
  \item 12V simulating an extremely high vibration (devastating earthquakes on the Richter scale, over 7 in magnitude)
\end{itemize}


\section{Results}
\label{sec:results}
Applications which monitor geodynamics using wireless sensor networks have been attempted before. However, most have been used to
monitor volcanic activity, tsunamis and building structure integrity but there are seldom any references of attempts to monitor 


\begin{figure}[ht] \centering
  \includegraphics[width=0.5\textwidth]{img/2v.png}
  \caption{Results for the shake table at 2V}
\end{figure}

\begin{figure}[ht] \centering
  \includegraphics[width=0.5\textwidth]{img/4v.png}
  \caption{Results for the shake table at 4V}
\end{figure}

\begin{figure}[ht] \centering
  \includegraphics[width=0.5\textwidth]{img/7v.png}
  \caption{Results for the shake table at 7V}
\end{figure}



\section{Conclusions and Future Work}
\label{sec:conclusion}
\subsection{Conclusions}

In this paper, through the previously presented experiment, we have shown 
that the Sparrow v4 wireless sensor node is a viable tool for monitoring 
earthquake activity. It is also versatile as it can fulfil multiple purposes 
and tasks. Sparrow v4 nodes can be placed in key points on a structure to monitor 
the amount of vibrations which it is exposed to. Alternatively, the nodes can 
be organized in a wireless sensor network and can be used to monitor seismic activity 
in certain areas. They can act as a tool which is used to predict earthquakes and 
warn authorities in case of danger.

The results have shown that the experimental rig is responding similar to a tall building because 
the lower positioned nodes gather lower amplitude values than the ones located on higher shelves 
when the table is vibrating. At the end of the simulation, when the table is no longer being shaken 
and it is naturally vibrating due to previous forces, the higher positioned nodes continue 
to detect vibrations, as it is most likely to happen in the case of a real building. Even more, the 
system could detect flaws in various structures and prevent real earthquakes from causing their collapse.

Furthermore, because the sensor nodes are low power, no matter in which type of application they 
are used, they will be able to function for prolonged periods of time, which makes them ideal 
for monitoring environments.

\subsection{Future Work}
The Sparrow v4 sensor nodes can be further improved by adding energy harvesting modules to their 
current design. Allowing them to harvest resources such as solar power can further increase 
their autonomy and efficiency. Further more, there is the possibility of designing a node which needs even less power to function. 
This can be achieved mainly by swapping the microcontroller with an Atmel SAM R21, which brings the following advantages compared to the existing one:

\begin{itemize}
    \item 32 bit architecture instead of 8 bit architecture.
    \item 256 kB of flash instead 128 kb of flash. 150k flash endurance instead of 50k , and 16k eeprom emulation with 600k write cycles endurance instead of 4k eeprom with 100k write cycles. This will allow more data to be saved more often and bigger programs to be used.
    \item 32 kb of sram instead of. This will allow for higher complexity programs and algorithms.
    \item Higher operating frequency, 48 MHz instead 16 MHz, while maintaining the same power consumption of 5 mA.
    \item 70 percent smaller footprint, 32 QFN instead of 64 QFN. This allows for smaller nodes that can be mounted more easily.
    \item Small improvement in transceiver's power efficiency and transmission power, 4 dB instead of 3.5 dB.
    \item 12 bit ADC instead of 10 bit ADC for improved sensor reading accuracy.
\end{itemize}

Another component which can be swapped is the accelerometer. There are models which have a consumption of only 6uA for a 50Hz ODR.
Howver, the trade-off for these changes would be a sight increase in the cost of the nodes.


\section{Acknowledgements}
\label{sec:Acknowledgements}
The work has been funded by the Sectoral Operational Programme Human Resources Development 2007-2013 of the Ministry of European Funds through the Financial Agreement POSDRU/159/1.5/S/134398.

\bibliographystyle{abbrv}
\bibliography{quake}

\end{document}
