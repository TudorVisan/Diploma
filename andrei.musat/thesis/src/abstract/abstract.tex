\chapter*{Abstract}

\vspace{\baselineskip}

Diagrams are used in a variety of fields such as hardware design and verification as a means to represent workflows and concepts. They require dedicated algorithms in order to produce easy to understand results. 
Currently available software is clunky and, at times, it can be really slow and will often result in layouts which are hard to read and understand. The main issue of layout algorithms is the sheer amount 
of data that a graph can contain, which makes placing and routing diagrams in a limited space a very difficult problem. We solve this problem by using a two phase algorithm which tries 
to generate layouts which feel natural for the user. The first phase will layout the diagrams using a genetic algorithm and a set of pre-determined constraints, while the second 
phase will route the paths connecting the diagrams elements using orthogonal routing. With this approach, we can provide layouts which suit the needs and preferences of any user, instead of providing 
generic representations.

\vspace{\baselineskip}
\textbf{Keywords} Diagrams, Layout Processing, Orthogonal Routing, Genetic Algorithms, Hardware Design
