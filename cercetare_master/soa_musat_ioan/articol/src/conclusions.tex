In conclusion, over the course of the experiment we have observed that the 
software implementation of ECB and CBC encryption algorithms can be used in low 
power wireless sensor networks as an alternative to the hardware AES.

While the method does have its drawbacks, the results show that it is an 
acceptable solution both in terms of power consumption efficiency as well 
as overall performance of the SparrowE node.

We have seen that using this implementation, the number of encryption 
and decryption operations which can be performed in the unit of time 
is significantly smaller than when using hardware AES. However, for the 
needs of the network, the number of operations is sufficient.

A second drawback would be that the software implementation requires 
additional memory to store the tables used by the ECB and CBC methods 
to generate keys. This can take up to 21Kb of flash memory on the controller.
In comparison, when using hardware AES, no such memory usage is required.

On the advantages of using this method, we have shown that the power consumption 
of the Atmega controller, when used to perform intensive computations, is not 
something that would decrease the sensor node's autonomy. Therefore, we can 
use this method safely in a low power network which must operate for increased periods 
of time.

The last test has shown that by using this method of decryption, the coordinator (or 
gateway) node can still perform even when flooded with a great number of corrupt 
packages meant to perform a DDoS attack and prevent the network from processing 
proper data.
