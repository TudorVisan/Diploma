Applications which monitor geodynamics using wireless sensor networks have been attempted before. However, most have been used to
monitor volcanic activity, tsunamis and building structure integrity but there are seldom any references of attempts to monitor 
earthquakes directly using wireless sensor networks.

Researchers from Singapore, China and the USA have published a paper describing their implementation of an improved algorithm 
for WSNs in order to monitor volcanic activity. This new algorithm, using data gather from the sensors, would determine as accurately as 
possible and in real time the arrival of primary seismic waves which are produced prior to a volcanic eruption. Although not 
directly focused on earthquake monitoring, this research provides a starting point for further research and improvements in 
this area.

Another implementation using WSNs is presented by N. Meenakshi and Paul Rodrigues and focuses on tsunami monitoring using WSNs. 
They propose a network composed of 3 types of nodes: sensors, commanders and barriers. The sensors are dispersed underwater 
in order to monitor the watter pressure. This data is sent to commanders which process it and determine if there is any specific area in danger 
of being hit by an incoming tsunami, due to variations in pressure. If there is any danger, the barrier sensors in 
that area are notified to activate the barriers.

One more direction in which geodynamics monitoring WSNs have been used is structural integrity of buildings. Especially in urban 
areas, buildings are often exposed to vibrations caused by various factors: heavy vehicles such as
public transport or cargo trucks, 
proximity to construction sites, etc. In time, such buildings deteriorate and become a danger because they are prone to collapsing. 
Using such sensor to monitor the vibrations they are exposed to, damage can be prevented by
determining if the building might collapse and put civilians in danger.
