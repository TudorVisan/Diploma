% ********** Abstract **********

\chapter*{Abstract}

This thesis proposes a new communication protocol for Wireless Sensor Networks
designed to take advantage of alternative energy sources. Despite improvements
in both energy efficiency and sensing capabilities Wireless Sensor Networks are
still limited by the finite battery life of their nodes. Similar protocols
focus on low energy consumption to the detriment of network bandwidth and
responsiveness and never take into account the actual environment around the
nodes. Our approach is based on harvesting and storing energy from the
surrounding environment in order to enable longer and more intense periods of
activity while preserving battery life. To this end we divide the network
nodes in two categories: leaf/measuring nodes, which run at very low
duty-cycles to conserve battery life, and root/routing nodes, which run at
higher duty-cycles in order to increase network bandwidth and responsiveness,
and which must pull additional resources from alternate energy sources. Through this
technique data loads within Wireless Sensor Networks may increase, thus
allowing more precise and complex measurements and greater network size, all
while preserving node lifespans.

\vspace{\baselineskip}

\textbf{Keywords} Wireless Sensor Networks, alternative energy, energy
harvesting, scheduling, synchronization

% ********** End of Abstract **********
