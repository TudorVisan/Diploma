% ********** Sparrow protocol **********

\chapter{Sparrow Protocol}

\todo{intro}
\todo{generalities}
\todo{slot time, guard time}
As mentioned earlier, the main focus of this protocol is to increase network
bandwidth and responsiveness. This is achieved through larger windows of
activity for nodes closer to the gateway. 

\section{Initialization and registering into the network}

After power-on all Sparrow nodes will first read their pre-programmed id from
\mbox{EEPROM} memory and then attempt to register to a network. No restrictions
such as network address or parent id are imposed. As such any distribution of
Sparrow nodes where all leaf nodes are within communication distance of at
least one root node and a path can be formed from any root node to the gateway
will eventually form a network. 

There are two major approaches to registering to a network: the new node can
request registration information from the existing nodes (request method) or
the existing nodes can periodically advertise this information for new nodes
(advertisement method). 

\subsection{Request method}

The first method implies that any new node must broadcast a discovery message,
wait for responses, choose the best parent node from the offers received and
finally register to that node. With this method registration can be
unsuccessful for a number of reasons. Firstly, there may be no root nodes
within communication distance of the new node, however, the most we can do is
set a timeout and, if this timeout is reached and no offer is received, place
the node in a sleep state and retry the procedure after a certain amount of
time. Secondly, there may be root nodes in range but none are awake or in
listening state. This can be partially solved by prolonging root node listening
periods, albeit to the detriment of power consumption. Thirdly, there may be
root nodes in listening mode and in range but the discovery message may be
corrupted by an already scheduled transmission. The possibility of this
happening can be reduced by spacing transmissions evenly within the root node's
listening period. Fourthly, considering that the discovery message was
successfully received, the offer message may be corrupted because two nodes
attempt to respond at the same time. This means that a contention period must
precede any response, lengthening the registering node's listening window and
requiring multiple attempts on the behalf of the root nodes in order to
respond. Lastly, considering that the discovery message was received and there
was at least one offer, the following message sequence, i.e. the actual
registration process, may be corrupted by a scheduled transmission or other
phenomena and the process will have to be repeated.

As can be seen this method implies low costs for the registering nodes as they
must only transmit the discovery message and listen a relatively short time for
offers and they must pay this cost only once, during the registering process.
Root nodes however must listen for long periods of time which implies a high
power cost which must be payed every network period. Furthermore, this method
does not scale well with the number of nodes in the network. The chances of the
discovery message interfering with a scheduled transmission increase with the
number of nodes in the network. The same applies if more nodes are attempting
to register at the same time. This means that as the network grows it will
become increasingly difficult for new nodes to join.

\subsection{Advertisement method}

The second method involves all root nodes periodically transmitting a beacon,
or Router Advertisement, that contains all the information necessary to
register to the network. New nodes must listen for these messages and then
communicate with their chosen parent in order to register.  There are as well
multiple issues that we must address with this procedure. Firstly, there may be
no root nodes in range of the new node so, as with the first method, we can set
a timeout for the registration process. Secondly, although there is at least
one root node in range, the request may interfere with a scheduled
transmission. To avoid this, we reserve a registration slot for each root node
in which it listens for incoming requests and no other communication is
scheduled. Thirdly, although the request may have been sent within the root
node's registration slot, it was corrupted by another node attempting to
register to the same root. This can be resolved by adding a contention slot
before the registration. A node must first listen for a random period of time
and then successfully transmit a message to the root node in the contention
period in order to begin the registration process. Finally, there is no
guarantee that there will be a victor after the contention process. Sparrowv3
nodes have a 17 μs transition delay between listening and transmitting states
which means that if two nodes are offset by less than this they cannot avoid
each other and a collision will happen. \todo{this can be resolved by waiting
in multiples of 17us or more} This cannot be resolved because two
nodes may choose the same listening period randomly.

This method implies a longer listening period for registering nodes however
this cost must only be paid once. The cost for root nodes is relatively low as
they must only transmit the advertisement message and must have two extra
listening slots. It also scales relatively well with the number of nodes in the
network.  If all contentions have a winner then the mean waiting time scales
proportionally to the number of nodes attempting to register with a root node. 

\vspace{\baselineskip}

As the comparison shows the advertisement method is the better choice for
Wireless Sensor Networks as the average cost of registration is lower and it
scales better with network growth. In our implementation new nodes do not
initially know the network period, so they must extract it from the first
Router Advertisement they receive. They then wait one network period to collect
advertisements, choose the best root node available as their parent and set a
wake-up timer for the contention slot. In the contention phase the node chooses
a random listening period that is at most the full contention slot. If another
transmission is received, the node goes back to sleep and retries registering
the next period. If there is no other transmission, it sends a registration
request in which it specifies whether it is a root or leaf node.  Leaf nodes
are allocated one transmission slot whereas root nodes are implicitly allocated
16. If this request is received correctly by the root node, i.e. a conflict has
not occurred, the root node acknowledges the registration and the new node
transitions into normal operating mode.

\section{Leaf node operation}
\label{sec:leaf_node_operation}

The main focus of leaf node operation is low power consumption. They are only
active during the communication slot with their parent. In this slot they
transmit a message to their parent and wait for an acknowledgement message. If
the acknowledgement does not arrive the frame is considered lost but it is not
retransmitted. If 10 transmissions is a row fail then the parent node is
considered to have gone off-line and the node tries to re-register to the
network. The SparrowLibrary allows the user to specify two additional
activities. The first is a function to be called after a frame is received .
The second is a function to be called before a frame is to be transmitted.
Through these functions the programmer may interact with the application frame
buffer. For the transmission callback the user must also specify how long
before the transmission window the function should be called. The programmer is
responsible for estimating the duration of the callback. If the delay is too
short the transmission timeout will interrupt the function and its execution
will be resumed after the transmission is started and may be interrupted again
when the acknowledge message is received.

\section{Root node operation}
\label{sec:root_node_operation}

The root nodes' responsibilities in the network are much more elaborate those
of the leaf nodes. Obviously, this comes with a higher duty cycle and power
consumption. 

Firstly, in the registration phase root nodes will be allocated 16 time slots
and they are responsible for allocating them to their children. In the case
that these slots become exhausted a root node can request another 16 time slots
from its parent. After registering, the root node will transmit one locally
generated message to its parent every network period and wait for an
acknowledgement.  This frame is sent during the 13\textsuperscript{th} time
slot of this block.  The 14\textsuperscript{th} and 15\textsuperscript{th}
slots are the contention and registration slots. Slots 1 through 12 are
reserved for receiving data from the node's children and transmitting this data
forward. For example, the data received from the node in slot 3 is sent forward
towards the gateway in slot 4.  This way only one frame must be buffered,
reducing the node's memory usage. The 16\textsuperscript{th} slot is reserved
for the Router Advertisement.

Secondly, the node must be able to allocate time slots for new nodes that wish
to join the network and existing root nodes that require more slots. In either
case time slots are allocated from the node's pool in descending order. There
are two moments when the node may be required to allocate new slots: during the
registration period and during the reception periods. In the registration slot
new nodes may requests to join the network and ask for one slot, if a leaf node
is registering, or 16 slots, if a root node is registering. In the reception
slots the sender node may include a request for more slots in the message
however, this request will only be taken into consideration after the
contention period so as to not hinder new nodes from joining the network. In
either case, the root node will accept the request if it has enough slots
available and deny it otherwise.  

Thirdly, root nodes must transmit Router Advertisements in order to allow new
nodes to join the network. The advertisement message contains the root node's
id, the network period and the number of free slots that the node has. The fact
that the advertisement is sent in the last time slot ensures that the number of
available slots will not change until the next contention period.

Lastly, if during 10 consecutive reception periods no message is received from
the node in question, the root node must mark the slot as free and advertise it
as such.

\section{Gateway operation}

The gateway is a crucial component of any Wireless Sensor Network. It acts as
an interface between the network and the end user as they typically forward
data from the network to a server or database. In our application the gateway
is the root node with id 0 and is always online. It is also the only node to
have the network period pre-programmed. All other nodes learn this information
from Router Advertisements.

The gateway initially owns all time slots within the network period and it
allocates slots to other nodes if requested. After a slot is allocated to a
node that node becomes the slot's owner and can use it as described in 
\ref{sec:leaf_node_operation} and \ref{sec:root_node_operation}.

In our implementation the gateway is represented by a SparrowDongle but may
also be a Sparrowv3 node attached to the programming dock.

% ********** End of Sparrow protocol **********
