In recent years, wireless sensor networks have been used more and more often as a cheap and easy to maintain monitoring system.
Wireless sensors are adequate tools for monitoring various environments due to a sers of characteristics such as: low power 
consumption which increases autonomy and helps reduce the node size (no need to attach large batteries), the possibility to 
attach energy harvesting modules which further help to increase their autonomy and the ability to mount numerous sensor peripherals 
on a small surface.

Since these sensors communicatie via wireless networks, they are ideal tools to use in monitoring remote or otherwise hostile 
environments such as: underground caverns, ocean floors, volcanic mountain ranges, etc. Another field in which wireless sensors 
would represent a good monitoring solution is geodynamics. Using such a netowrk, it could be possible to predict phenomena such 
as earthquakes, tsunamis, landslides and volcanic eruptions much faster and with increased precission regarding magnitude and the 
time of the event. Also, such a solution may prove to be cheaper and more flexible than existing installations deployed for performing 
these tasks.

Delving even deeper into the utility of such an aplication, we can take into account that whenever a phenomena amongst those mentioned 
earlier occurs, the areas which often suffer damages are cities and towns. A good use for a wireless sensor network would be to mount it 
around cities and on buildings inside the city situated in such danger zones. Thus, whenever an earthquake or other natural disaster occurs, 
authorites can respond faster and damage, both material and to human lives, can be reduced.

In this paper, we propose a wireless sensor network solution for monitoring earthquakes. The nodes shall be mounted on buildings and 
they shall monitor the vibration of the building as well as various other parameters (air pressure, temperature, etc.). Once calibrated 
to the "normal" values of the parameters, whenever these parameters go over a threshold value, a system is notified that there is a possible 
risk of an earthquake happening. More so, this system can be used to determine if a building is exposed to deterioration due to promximity 
to construction site, roads frequented by heavy load trucks.
