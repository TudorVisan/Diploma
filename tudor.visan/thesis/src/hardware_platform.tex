% ********** Hardware platform **********

\chapter{Hardware platform}

This protocol was developed and tested entirely on the Sparrowv3 wireless
sensing platform. This platform was developed by Sl. Dr. Ing. Dan Tudose, As.
Drd. Ing. Andrei Voinescu and As. Drd. Ing. Dan Dragomir at the Faculty of
Automatic Control and Computers, University Politehnica of Bucharest. The
development kit consists of Sparrowv3 sensing nodes and the SparrowDongle usb stick.

\color{red} 
// TODO: get actual list of authors
\color{black}

\begin{figure}[ht]
	\begin{center}
		\includegraphics{img/placeholder.jpg}
	\end{center}
	\caption{\small \itshape{The Sparrowv3 development kit}}
\end{figure}

Together, the sensing nodes and usb dongle form a tool for developing and
debugging applications that are IEEE 802.15.4 and ZigBee compatible. Such
applications range from environmental monitoring and data logging to wildlife
research.

\section{The Sparrowv3 nodes}

The Sparrowv3 wireless sensing nodes are small, battery powered devices capable
of IEEE 802.15.4 and ZigBee compatible communication. At the center they have
an \mbox{ATmega128RFA1}, a microcontroller made by Atmel which is based on the
\mbox{ATmega128}. It incorporates a powerful and energy-efficient AVR
core and comes equipped with all the standard AVR peripherals such as multiple 
8 and 16-bit synchronous timers, an 8-bit asynchronous timer, SPI (Serial
Peripheral Interface), USART (Universal Synchronous Asynchronous Receiver
and Transmitter) and TWI (Two Wire serial Interface) communication interfaces 
and an ADC (Analog-to-Digital Converter). The controller offers an advanced power 
management interface and sleep modes. This permits the user to deactivate 
peripherals and put the CPU into varied sleep modes in order to preserve 
power. In its most efficient sleep state the microcontroller utilises only 
0.25 uA of current (at a supply voltage of 3V). The controller is clocked from an
external 16 MHz crystal oscillator.

\begin{figure}[ht]
	\begin{center}
		\includegraphics{img/placeholder.jpg}
	\end{center}
	\caption{\small \itshape{A Sparrowv3 sensing node (front and back)}}
\end{figure}

The \mbox{ATmega128RFA1} has 4K bytes of EEPROM memory used to hold
write-protected configuration and calibration data, 128K bytes of In-System
Self-Programmable Flash memory, which is used to store the firmware and 16K
bytes of SRAM, used for storing data during execution.

The microcontroller also incorporates a low-power 2.4 GHz transceiver and a MAC
symbol counter. The transceiver has a maximum sensitivity of -100 dBm and
includes a 128-bit hardware AES accelerator and a 2-bit true random number
generator. An external antenna and a balun must be connected to the
controller and special precautions must be taken when designing the PCB trace
for them in order to ensure optimal functionality. The MAC symbol counter is 
a 32-bit counter that provides the timing information for the transceiver. 
The counter runs by deriving a 62.5 kHz clock from a 32.768 kHz low-drift 
crystal oscillator. 

\color{red}
// TODO: power supply

// TODO: external connectors

// TODO: sensors
\color{black}

\section{The SparrowDongle}

The SparrowDongle description.

\begin{figure}[ht]
	\begin{center}
		\includegraphics{img/placeholder.jpg}
	\end{center}
	\caption{\small \itshape{The SparrowDongle (front and back)}}
\end{figure}

% ********** End of Hardware platform **********
