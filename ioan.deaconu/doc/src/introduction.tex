\normalfont\normalsize
\chapter{Introduction}

Technology has been present in our lives for a long time and we have become so accustomed to it that we do not even realize when we are using it.

This technological advance has lead to inventing small devices with low power consumption and communication capabilities. They form a wireless sensor network and can be used to collect data from the environment and send them to a gateway for further processing. In a way, even the new gadgets that have appeared this year, gadgets like smart watches and fitness trackers, can be seen as wireless sensor devices because they collect data like steps, heart rate and have wireless capabilities.

The applications of a Wireless Sensor Network are unlimited in number. For example they can be used to monitor crops, to detect possible forest fires, to detect the presence of animals or vehicles in certain areas, tracking and monitoring doctors in hospitals, guidance interactive toys, detecting and monitoring car thefts etc. In the last ten years, integration of wireless sensor networks with unmanned aerial vehicles had been tested and proven to be successful. So far, the proposed implementations, described in chapter \ref{chap:related}, are complicated, difficult to operate and too expensive for the general public.

The goal of Wireless Sensor Networks is to collect data and send it to a device called gateway. The scenario that can be applied to the real world is that of a user that  has a number o nodes deployed in an area. He can not reach that area for certain reasons, but it can send an unmanned aerial vehicle there. The way the UAV reaches that area can be either by being controlled by the user through a remote the entire time or to follow a list of way points predetermined by the user. When the UAV reaches the nodes, it will save the data muled from the nodes and send back some or all of the gathered information. The data collected by the drone should be accessible at any time, if the UAV is powered.

 
The solution that we propose is based on a very popular, easy to use AR Parrot Drone 2.0 and the Sparrow Family \cite{voinescu2013lightweight}, developed at Polytechnic University of Bucharest by Andrei Voinescu.
