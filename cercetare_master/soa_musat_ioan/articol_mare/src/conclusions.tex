\subsection{Conclusions}

In this paper, through the previously presented experiment, we have shown 
that the SparrowE wireless sensor node is a viable tool for monitoring 
earthquake activity. It is also versatile as it can fulfill multiple purposes 
and tasks. SparrowE nodes can be placed in key points on a structure to monitor 
the amount of vibrations which it is exposed to. Alternatively, the nodes can 
be organized in a wireless sensor network and can be used to monitor seismic activity 
in certain areas. They can act as a tool which is used to predict earthquakes and 
warn authorities in case of danger.

Furthermore, because the sensor nodes are low power, no matter in which type of application they 
are used, they will be able to function for prolonged periods of time, which makes them ideal 
for monitoring environments.

\subsection{Future Work}
The SparrowE sensor nodes can be further improved by adding energy harvesting modules to their 
current design. Allowing them to harvest resources such as solar power can further increase 
their autonomy and efficiency. Improvements can also be made to the experimental rig. A collaboration 
with experts in seismology would be ideal in order to properly calibrate the rig to simulate 
proper earthquakes on the Riechter scale and produce more accurate movements.
