\normalfont\normalsize
\chapter{Dynamic Scheduling}

The dynamic version of the algorithm can burrow heavily from the static one. The need for dynamics arises from the interface
with outside components that will control some aspects of the wireless application running. Even if the tasks are already present
on the wireless nodes as chunks of code, starting and stopping them could be arbitrarily decided from outside the network.
In this case, the speed with which the new task is assigned or the rescheduling takes place gives the responsiveness of the network,
which can be crucial in some respects.

We will propose two ways of assigning new tasks, one is a greedy approach and the other is based on an update of the model 
described and calculated with the static scheduling algorithm. The former will be a lot faster, but will differ from the
optimum solution.

\section{On-the-fly Dynamic Scheduling}

The dynamic problem is inherently that of starting a new task. In the on-the-fly method, stopping tasks holds no rescheduling
consequences. Even with using a Greedy approach, the algorithm can obtain good results is a good minimization function is found.
The purpose is still to maintain an extended network lifetime. If a node has to receive a new task, it could be the one with the most
network lifetime:
\begin{equation}
 M(v_{new}) = \max_{k,\,m_k\in M} Lifetime(m_k) 
\end{equation}
\begin{equation}
 M(v_{new}) = \max_{k,\,m_k\in M} \frac{W_{m_k}}{\displaystyle W_{idle_{m_k}} +  W_{communication_{m_k}}}
\end{equation}

However, this would not be the best approach because the node with the longest remaining lifetime might not be compatible, or dependencies
of the new task could be quite costly. The minimization function must take into account the impact on the node's communication energy 
consumption.

\begin{equation}
 f(v_{new},m_k) = \min_{k,\,m_k\in M} \frac{\displaystyle \frac{1}{L(m_k)} + \alpha \cdot (L'(m_k) - L(m_k))}{NA(v_{new},m_k)},
\end{equation}
where:
\begin{itemize}
 \item $L(m_k)$ - The lifetime of node $m_k$.
\item $L'(m_k,v_{new})$ - The lifetime of node $m_k$ after $v_{new}$ is assigned.
\item $NA(v_{new},m_k)$ - The affinity of the new task to the node $m_k$.
\item $\alpha$ - A balancing coefficient between keeping the greatest network lifetime (picking the node with the most battery) 
and network-level energy saving (choosing a node for our new task such that extra communication is minimized).
\end{itemize}


\section{Recalculating the static schedule}

Another variant of dynamic scheduling is based on the approximation algorithm. Gomory-Hu trees can be recalculated quite fast, 
as the s-t cuts do not change much with the addition of an extra edge (the atomic operation is considered here to be adding
an edge, adding a node to a DAG does not change anything if it is not connected). Some of the computations can be retained
so that calculations can resume right where the new edge was inserted and finish quite quickly, even faster than the normal 
approximation algorithm.